\makeatletter

\usepackage{datatool, fp}

\def\showevColorOne{blue!50!black}
\def\showevColorHigh{blue!60}
\def\showevColorLow{blue!20}
\def\showevColorZero{white}

\def\showeval@basecolor{blue!70!black}

\newlength{\showeval@xshift}\setlength{\showeval@xshift}{5.5mm}
\newlength{\showeval@nodewidth}\setlength{\showeval@nodewidth}{5.5mm}
\newlength{\showeval@yshift}\setlength{\showeval@yshift}{4.5mm}
\newlength{\showeval@nodeheight}\setlength{\showeval@nodeheight}{3.5mm}

%%%%%%%%%%%%%%%%%%%%%%%%%%%%%%%%%%%%%%%%%%%%%%%%%%

\DTLnewdb{showeval@db}

\newcounter{showeval@counI}
\newcounter{showeval@counII}

%%%%%%%%%%%%%%%%%%%%%%%%%%%%%%%%%%%%%%%%%%%%%%%%%%

\def\showeval@drawticks{
    \foreach \showeval@point in {0, 1, 2, 3, 4, 5, 6, 7, 8, 9, 10} {
        \node[
            rectangle, \showevColorHigh,
            xshift=\showeval@point\showeval@xshift
        ]{\small\showeval@point};
    }
}

\def\showeval@drawticks@rough{
    \node[
        rectangle, \showevColorHigh,
        xshift=1\showeval@xshift,
        minimum width=3\showeval@xshift,
        text height=3mm, text depth=1mm
    ]{\small плохо};
    \node[
        rectangle, \showevColorHigh,
        xshift=5\showeval@xshift,
        minimum width=5\showeval@xshift,
        text height=3mm, text depth=1mm
    ]{\small средне};
    \node[
        rectangle, \showevColorHigh,
        xshift=9\showeval@xshift,
        minimum width=3\showeval@xshift,
        text height=3mm, text depth=1mm
    ]{\small хорошо};
}

\def\showeval@drawonebar@fullscale@fullrange#1{
    \foreach \showeval@point in {0, 1, 2, 3, 4, 5, 6, 7, 8, 9, 10} {
        \setcounter{showeval@counII}{\showeval@point+1}
        \def\showeval@pl{\csname showeval@p\Roman{showeval@counII}\endcsname}
        \FPeval{\showeval@density}{round(round(\showeval@pl * 10, 0) * 10, 0)}
        \node[
            rectangle, draw,
            minimum width=\showeval@nodewidth, minimum height=\showeval@nodeheight,
            white, fill=\showeval@basecolor!\showeval@density, draw=black,
            xshift=\showeval@point\showeval@xshift,
            yshift=#1
        ]{};
    }
}

\def\showeval@drawonebar@discrete@fullrange#1{
    \foreach \showeval@point in {0, 1, 2, 3, 4, 5, 6, 7, 8, 9, 10} {
        \setcounter{showeval@counII}{\showeval@point+1}
        \def\showeval@pl{\csname showeval@p\Roman{showeval@counII}\endcsname}
        \FPeval{\showeval@density}{round(\showeval@pl *3, 0)}
        \ifthenelse{\equal{\showeval@density}{0}}{\def\showeval@cur@color{\showevColorZero}}{}
        \ifthenelse{\equal{\showeval@density}{1}}{\def\showeval@cur@color{\showevColorLow}}{}
        \ifthenelse{\equal{\showeval@density}{2}}{\def\showeval@cur@color{\showevColorHigh}}{}
        \ifthenelse{\equal{\showeval@density}{3}}{\def\showeval@cur@color{\showevColorOne}}{}
        \node[
            rectangle, draw,
            minimum width=\showeval@nodewidth, minimum height=\showeval@nodeheight,
            white, fill=\showeval@cur@color,
            xshift=\showeval@point\showeval@xshift,
            yshift=#1
        ]{};
    }
}

\def\showeval@display#1#2#3{
    \DTLsetseparator{ }
    \DTLnewdbonloadfalse
    \DTLloaddb[
        noheader, 
        keys={p00,p01,p02,p03,p04,p05,p06,p07,p08,p09,p10}
    ]{showeval@db}{#1}
    \DTLnewdbonloadtrue
    \begin{tikzpicture}
        \ifthenelse{\equal{#3}{rough}}{\showeval@drawticks@rough}{\showeval@drawticks}
        \setcounter{showeval@counI}{0}
        \DTLforeach{showeval@db}{
            \showeval@pI   =p00,
            \showeval@pII  =p01,
            \showeval@pIII =p02,
            \showeval@pIV  =p03,
            \showeval@pV   =p04,
            \showeval@pVI  =p05,
            \showeval@pVII =p06,
            \showeval@pVIII=p07,
            \showeval@pIX  =p08,
            \showeval@pX   =p09,
            \showeval@pXI  =p10%
        }{
            \stepcounter{showeval@counI}
            \ifthenelse{\equal{#2}{full scale}}{%
                \showeval@drawonebar@fullscale@fullrange{-\value{showeval@counI}\showeval@yshift}%
            }{%
                \showeval@drawonebar@discrete@fullrange{-\value{showeval@counI}\showeval@yshift}%
            }
        }
    \end{tikzpicture}
    \DTLgcleardb{showeval@db}
}

\def\showevDisplayFullScale#1{\showeval@display{#1}{full scale}{full range}}
\def\showevDisplay#1{\showeval@display{#1}{discrete}{full range}}
\def\showevDisplayRough#1{\showeval@display{#1}{discrete}{rough}}

\makeatother
