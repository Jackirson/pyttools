
\begin{frame}{Выводы работы}
	\begin{itemize}
	 \item 
	 Поставлена и решена задача принятия решения о выборе наиболее качественных объектов на основе нечётких экспертных данных об их характеристиках, сформулированная как задача на минимум возможности принять неверное 
	 решение (ошибиться);	
	\item 
	Разработан численный метод, позволяющий решить эту задачу за время порядка $O(nm)$ вместо $O\big(\abs{X}^{nm}\big)$ в случае полного перебора всех элементраных событий. %, где $n$ --- количество объектов, $m$ --- количество параметров у каждого объекта, а $\abs{X}$ --- размер множества $X$ значений параметров.
	\item
	Поставлена и решена задача нахождения коллективного мнения экспертов как задача нахождения верхней грани на множестве возможностных распределений (экспертных оценок), используя введённое на нём отношение квазипорядка.
	\item
	Разработан численный метод решения этой задачи за время порядка $O\big(R(nm)^2\big)$ вместо $O\big(R\abs{X}^{2nm}\big)$ в случае построения совместного распределения возможностей всех параметров всех объектов. %, где $n,m,\abs{X}$~--- те же, что выше, а $R$ --- количество экспертов. 
      \end{itemize}
\end{frame} %===========================

%\begin{frame}{Выводы (задача выбора объектов)}
%	\begin{itemize}
		
%	\end{itemize}
%\end{frame} %===========================

\appendix

% === final slide ====
\begin{frame}{Спасибо за внимание!}
	\begin{center}
		\includegraphics[width=0.5\linewidth]{./pic/biber_final}
	\end{center}
\end{frame}
% =============

\begin{frame}{Метод векторов предпочтений}
	\vspace{-3mm}
	{ \small Коллективное мнение экспертов с помощью векторов предпочтений аналогично случаю матриц попарных сравнений ($r = 1 \ldots R$, $J = \{1\ldots\abs{T}+1\})$:}
	\begin{gather*}
		s^{(r)}_j = \sum_{t \in T} \mathlarger{\mathlarger{\chi} }_{A_j^{(r)}}(t),
		 \mathsmaller{\text{\small причём } \p^{(r)}(t_{\abs{T}+1}) \define= 0, \; j \in J. } 
		 \\ \text{Здесь } A^{(r)}_j = \{t \in T: \p^{(r)}(t) \leq \p^{(r)}(t_j)\}. 
	\end{gather*}
	Вектор предпочтений $s$ должен удовлетворять условию:
	%\\ (1) $\max s_j= |T|$ (нормировка возможности);
	\\[1.2ex] ($\star$) $|J_i| = i$, где $J_i = \{j \in J: s_j \leq i\}$, $i \in \{s_1 \ldots s_{\abs{T}+1}\}$.
	
	\textbf{Задача}: $\displaystyle s_* = \arg \underset{s} \min \sum_{r=1}^R \rho(s - s^{(r)})$, где  $\displaystyle \rho(s, s') = \big( \sum_{j=1}^{\abs{T}+1}(s_{j} - {s'}_{j})^2 \big)^{1/2}$.
	
	\textbf{Алгоритм решения}:  берём  вектор $ \ol s =  \frac{1}{R} \sum_{r=1}^R s^{(r)}$ и удовлетворяем условию ($\star$) с помощью округления и (если требуется) дополнительного изменения его координат.
\end{frame} %===========================
