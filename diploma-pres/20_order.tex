\section{Задача и алгоритм коллективной экспертизы}

\begin{frame}{Задача коллективной экспертизы: варианты}
	\begin{enumerate}
		\item Методы Ю.~П.~Пытьева: матрицы попарных сравнений и др.;
		\item Новый метод для т.\,в.~Пытьева -- вектора предпочтений;
		\item Новый метод -- введение отношения препорядка (частичного порядка с точностью до эквивалентности) на множестве распределений нечёткого элемента, вычислние точной верхней/нижней грани распределений.
	\end{enumerate} 
\end{frame}

\begin{frame}{Вектора предпочтений}
\end{frame}

\begin{frame}{Препорядок распределений возможности}
	\emph{Опр.} $\p_1 \prec \p_2$, если:
	\begin{enumerate}
		 \item $\supp\;p_2 \supset \supp \; \p_1$

		  \item $\exists \gamma: \p_2(\omega) = \gamma(\p_1(\omega))
		   \omega \in \supp\;\p_1$ 

		  \item $p_2(\omega) \leq p_2(\omega'), \omega \not\in  \supp \; \p_1, 
		  \omega' \in  \supp \; \p_1$
	\end{enumerate}
\end{frame}

\begin{frame}{Алгоритм коллективной экспертизы}
\end{frame}