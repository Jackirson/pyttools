\section{Задача и алгоритм коллективной экспертизы}

\begin{frame}{Задача коллективной экспертизы: варианты}
	\begin{enumerate}
		\item Методы Ю.~П.~Пытьева\footnote{Пытьев\;Ю.\,П. \emph{Эмпирическое восстановление мер возможности и правдоподобия возможности в моделях экспертных решений}, 2009}:
% .}~// Докл. АН СССР, Т.\;224, \No 6, С.\;1283--1286, 2009.\\}: 
		матрицы попарных сравнений и др.;
	%%	\item Новый метод для т.\,в.~Пытьева -- вектора предпочтений;
		\item Новый метод -- введение отношения предпорядка (частичного порядка с точностью до эквивалентности) на множестве распределений нечёткого элемента, вычислние (точной) верхней/нижней грани распределений.
	\end{enumerate} 
	
	{ \small Коллективное мнение экспертов с помощью матриц попарного сравнения \\ ($R$ экспертов, 1 объект с 1 параметром): 
	\begin{columns}
	   \column{0.5\textwidth}
	      \begin{gather*}
		  {(m^{(r)})}_{kj} = \begin{cases}
			\;\;\;1,\;\; \p^{(r)}(x_k) > \p^{(r)}(x_j)\\
			\;\;\;0,\;\; \p^{(r)}(x_k) = \p^{(r)}(x_j)\\
			-1, \;\; \p^{(r)}(x_k) < \p^{(r)}(x_j)
		  \end{cases} 
		  \\  r = 1 \ldots R;\; k,j = 1\ldots\abs{X}+1; 
		  \\ \p^{(r)}(x_{\abs{X}+1}) \define= 0.  
	      \end{gather*}
	   \column{0.5\textwidth}
	     \vspace*{-5mm}
	      \begin{gather*}
		  m_* = \arg \min_m \sum_{r=1}^{R} \rho(m^{(r)}, m),
		  \\ \text{где } \rho(m, m') = \big( \sum_{k,j=1}^{\abs{X}}(m_{kj} - {m'}_{kj})^2 \big)^{1/2}.
		  \\ \text{$m_*$ не всегда просто найти!}
	      \end{gather*}
	\end{columns}  } 
\end{frame}

\begin{frame}{Вектора предпочтений}
	
	{ \small Коллективное мнение экспертов с помощью векторов предпочтений \\ ($R$ экспертов, 1 объект с 1 параметром), аналогично случаю матриц п.\,с.:}
	\begin{gather*}
		{s^{(r)}}_j = \sum_{x \in X} \mathlarger{\mathlarger{\chi} }_{A_j}(x): \;\;
		 \mathsmaller{ \p^{(r)}(x_{\abs{X}+1}) \define= 0; \;  r = 1 \ldots R;\; j = 1\ldots\abs{X}+1 }; 
		 \\ \text{здесь } A^{(r)}_j = \{x \in X: \p^{(r)}(x) \leq \p^{(r)}(x_j)\} 
	\end{gather*}
	Вектор предпочтений $s$ должен быть нормирован: $\max s_j= |X|$ и \emph{эквилизован}: число координат $|J|: s_j \leq s_0, j \in J$ должно быть не больше $s_0$.
	
	\textbf{Задача}: $\displaystyle s_* = \arg \underset{s} \min \sum_{r=1}^R \rho(s - s^{(r)})$, где  $\displaystyle \rho(s, s') = \big( \sum_{j=1}^{\abs{X}}(s_{j} - {s'}_{j})^2 \big)^{1/2}$.
	
	\textbf{Алгоритм решения}:  берём  $ \ol s =  \frac{1}{R} \sum_{r=1}^R s^{(r)}$, нормируем и \emph{эквилизуем}, изменяя как можно меньше его координат на как можно меньшую величину.
\end{frame}

   % \begin{tikzpicture}
  %    \begin{axis}[
%	width = 0.5\textwidth, height = 0.4\textwidth,
 %   xlabel = {$L$},
%    ylabel = {$\alpha$},
	%ymin = 0, ymax = 0.4,  % if left default there will be margins
	%xmin = 2, xmax = 30,
 %   xtick = {2, 6, 10, 14, 18, 22, 26, 30}
%	xticklabels = {,,},
%	yticklabels = {,,}
%	]
%	\addplot[blue, very thick] table[x=n, y=cone] {./pic/results.txt};
%	\addplot[dash pattern=on 6pt off 3pt on 1pt off 3pt] table[x=n, y=svm] {./pic/results.txt};
%     \end{axis}
 %   \end{tikzpicture}


\begin{frame}{Предпорядок распределений возможности}
	\begin{columns}
	  \column{0.6\textwidth}
	    \emph{Опр.} $\p_1 \prec \p_2$ (читается: <<уточняет>>), если:
	    \begin{itemize}
		 \item $\supp\;p_2 \supset \supp \; \p_1$;

		  \item $\exists \gamma: \p_2(\omega) = \gamma(\p_1(\omega))
		   \omega \in \supp\;\p_1$; \begin{center}{\footnotesize $\gamma: \zo \rightarrow \zo$ -- монотонная непрерывная, $\gamma(0)=0$, $\gamma(1)=1$} \end{center}

		  \item $p_2(\omega) \leq p_2(\omega'), \omega \not\in  \supp \; \p_1, 
		  \omega' \in  \supp \; \p_1$.
	    \end{itemize}
	    
	    \vspace*{3mm}
	    \emph{Опр.} $\check{\p} = \p_1 \vee \p_2$ (<<супремум>>), если
	    \vspace*{-2mm}
	    \begin{gather*}
		  \check{\p} \succ \p_1, \check{\p} \succ \p_2, \text{но: } \\ \forall\;\p' \neq \check{\p}, \p' \succ \p_1, \p' \succ \p_2: \p' \succ \check{\p}.
	    \end{gather*}
	    
	    \vspace*{2mm}
	    \emph{Опр.} $\hat{\p} = \p_1 \wedge \p_2$ (<<инфинум>>) -- аналогично.
	     
	     \vspace*{1mm}
	     \begin{center}
		  Итого, распределения одного и того же н.\,э. образуют полурешётку (справа).
	     \end{center}
	   \column{0.5\textwidth}
	     \begin{center}
	     	   triv
	     	    \vspace{5mm}
		\\ sup
	    \end{center} 
	     \begin{columns}
		  \column{0.5\linewidth}
		   \begin{center}
		     PP1
		     \vspace{5mm}
		  \\ P1
		   \end{center} 
		  \column{0.5\linewidth}
		   \begin{center}
		     PP2
		      \vspace{5mm}
		  \\ P2 
		   \end{center} 
	      \end{columns}
	      
	      
	\end{columns}
\end{frame}



\begin{frame}{Алгоритм коллективной экспертизы}
 \begin{center}
    Пусть имеется $n$ объектов с $m$ параметрами, и $R$ экспертов, каждый из которых оценил все объекты по всем параметрам (если это не так, отсутствующие распределения считаем тривиальными.)
	\\ \vspace{3mm} \textbf{Идея}: 
	\begin{enumerate}
		\item составить <<длинные>> вектор $(\p_{11}^{(1)}, \ldots \p_{nm}^{(1)});\; (\p_{11}^{(2)}, \ldots \p_{nm}^{(2)})$ и т.\,д.;
		\item рассчитать коллективный вектор $(\p_{11}, \ldots \p_{nm})$ методом $\vee$ или методом матриц п.\,с.;
		\item Одномерные <<куски>> этого вектора, $\p_{ij}$,  считать коллективным мнением экспертов по параметрам объектов {\footnotesize $i = 1 \ldots n$, $j = 1 \ldots m$}. 
	\end{enumerate}	
 \end{center}
\end{frame}