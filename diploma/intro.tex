
\subsection{Принятие решений в условиях неопределённости}
\label{sec:basic_intro}

Слово <<решение>> в русском языке имеет несколько оттенков. В математике чаще всего требуется строгая постановка задачи, после чего решение задачи, в зависимости от её свойств, может быть найдено аналитически или численно, случайно или не совсем случайно подобрано, получено с использованием дополнительных соображений,  априорной информации и с помощью анализа фактов. Решение может быть единственным, а может не быть, или не существовать вовсе. В двух последних случаях часто можно переформулировать задачу так, чтобы решение всё же существовало и было единственным хотя бы с точностью до эквивалентных решений. 

В более широком смысле, в практических и прикладных задачах, решение часто требуется не только найти, но и {\sl принять} (иногда --- отвергнуть). Принятие решения означает принятие человеком ответственности за правильность решения, будь оно решением математической задачи или нет. Человека, берущего на себя эту ответственность, будем называть {\sl лицом, принимающим решение}, а действия и средства, помогающие ему в этом ---  поддержкой принятия решения. Решение экстремальной задачи в математике называют оптимальным, и мы используем это слово в названии настоящей работы. Но следует помнить, что найденное оптимальное решение может быть неправильным в более широком (практическом) смысле, с точки зрения лица, принимающего решение. 

Во всех сферах человеческой деятельности информация играет роль, важность которой трудно преувеличить. В научной и деловой среде часто предпочитают максимально объективную информацию о предметах исследования, в качестве которых выступают некие объекты, процессы и явления. Например, объективной сч считается информация, полученная с помощью измерительных приборов и в результате анализа уже свершившихся фактов, если точность полученного результата достаточно высока, а сам результат можно воспроизвести в повторном исследовании. 

На основе объективной информации часто можно сразу рассчитать оптимальное решение или критерий принятия решения. Но такая информация есть в наличии не всегда. Существуют задачи, требующие принятия решения в условиях неопределённости --- когда одних лишь объективных данных о предмете исследования не хватает для поиска решения, оптимального в некотором смысле, и нет времени или даже принципиальной возможности их  получить. Например, можно говорить о работе в условиях неопределённости, если верно одно из следующих утверждений:
\begin{itemize}
 \item отсутствует фактическая информации о предмете исследования за достаточно продолжительный период времени (статистические данные);
 \item нет возможности количественного моделирования всех факторов, оказывающих существенное влияние на принятие решения, в наличии имеется только информация, отражающей качественную сторону явлений; 
 \item исследуется процесс, направление развития которого нетривиальным образом зависит от ещё не случившихся событий, которые могут случиться или не случиться в будущем;
 \item исследуется качественно новое явление или объект в процессе их развития, который уникален.
\end{itemize}

Во многих подобных случаях важную роль для принятия решения играет суждение, высказанное человеком --- экспертом, специалистом. Такое суждение, вообще говоря, не является объективным, поэтому будем обозначать его одним из следующих эквивалентных словосочетаний:
 \begin{itemize}
	\item субъективное суждение;
	\item экспертное мнение;
	\item ответ эксперта (на заданный ему вопрос). 
 \end{itemize}

Та или иная процедура принятия решения может считаться оптимальной в практическом смысле. Не все они интересны с математической точки зрения. Например, если предметы исследования находятся в той предметной области, где лицо, принимающее решение, достаточно компетентно, то оно <<само себе эксперт>> и для принятия решения не требуется математическая формализация субъективных суждений. Но если предметы исследования не покрываются одной конкретной предметной областью, а находятся в самых разных предметных областях, то лицу, принимающему решение, скорее всего, потребуется  мнение более компетентных экспертов. В этом случае может не потребоваться, а может и потребоваться математическая формализация и анализ ответа экспертов. %Тогда привлекается эксперт или коллектив экспертов, внешних по отношению лицу, принимающему решение.
% к этому абзацу нужны ссылки на другие примеры стратегий выбора в литературе (Миша?)
 
Приглашённый эксперт может высказать суждение как самопроизвольно, так и в ответ на вопросы, специально сформулированные для поддержки принятия какого-либо решения. Процесс подготовки вопросов и прочих материалов, получения и последующего анализа экспертных мнений будем называть {\sl экспертным опросом} или {\sl экспертизой}. Лицо, принимающее решение, или субъект, от имени которого действует это лицо, выступает здесь в роли {\sl заказчика экспертизы}.
 
Экспертное мнение может быть выражено в виде развёрнутых рекомендаций и заключений, но этот случай не представляет интереса с математической точки зрения и в настоящей работе не рассматривается. Нас интересуют ситуации, когда можно выбрать некоторые числовые параметры, характеризующие важные для принятия решения аспекты предмета (или предметов) исследования, а эксперта (или экспертов) просят дать математическую оценку того или иного параметра. Ответ эксперта в этой ситуации логично назвать {\sl экспертной оценкой}, а всевозможные математические модели экспертных оценок будут рассмотрены ниже.  
 
Надо отметить, что если задача принятия решения в условиях неопределённости имеет научно-технический характер, то иногда её решение уже назревает в мозгу специалистов, работающих в соответствующей области. Однако, это решение может быть ещё не оформлено в виде мыслей, имеющих достаточную чёткость для выражения. Экспертный опрос помогает осознать и формализовать эти мысли, после чего на суд лица, принимающего решение, выносятся не просто экспертные мнения по изначально предложенной специалистам схеме, а готовые варианты решения. 

\subsection{Цель работы}
Цель настоящей работы --- исследовать новый способ проведения экспертизы с применением высокоинформативных, но при этом хорошо формализованных и математически строгих экспертных оценок, которые моделируются методами теории возможностей Ю.~П.~Пытьева; развить методы теории возможностей Ю.~П.~Пытьева за пределами их классического изложения в монографиях и статьях Ю.~П.~Пытьева; поставить задачи, характерные для области поддержки принятия решений с использованием экспертных оценок; разработать эффективные алгоритмы и комплекс программ для решения этих задач. 

