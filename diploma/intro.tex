
\subsection{Основные термины}

Слово <<решение>> в русском языке имеет несколько оттенков. В математике чаще всего требуется строгая постановка задачи, после чего решение задачи, в зависимости от её свойств, может быть найдено аналитически или численно, случайно или не совсем случайно подобрано, получено с использованием дополнительных соображений,  априорной информации и с помощью анализа фактов. Решение может быть единственным, а может не быть, или не существовать вовсе. В двух последних случаях часто можно переформулировать задачу так, чтобы решение всё же существовало и было единственным хотя бы с точностью до эквивалентных решений. 

В более широком смысле, в практических и прикладных задачах, решение часто требуется не только найти, но и {\sl принять} (иногда --- отвергнуть). Принятие решения означает принятие человеком ответственности за правильность решения, будь оно решением математической задачи или нет. Такого человека будем называть {\sl лицом, принимающим решение}, а действия и средства, помогающие ему в этом ---  {\sl поддержкой принятия решения}. 

Решение экстремальной задачи в математике называют оптимальным, и мы используем это слово в названии настоящей работы. Но следует помнить, что найденное оптимальное решение может быть в более широком смысле, с точки зрения лица, принимающего решение, неправильным. 

Во всех сферах человеческой деятельности информация играет роль, важность которой трудно преувеличить. В научной и деловой среде часто предпочитают максимально объективную информацию, например, полученную с помощью измерительных приборов и в результате анализа уже свершившихся фактов. Но такая информация есть в наличии не всегда. Существуют задачи, связанные с исследованием сложных объектов, процессов и явлений, и требующие принятия решения в условиях неопределённости --- когда одних лишь объективных данных о предмете исследования не хватает для поиска решения, оптимального в некотором смысле, и нет времени или даже принципиальной возможности их  получить, например: 
\begin{itemize}
 \item при отсутствии фактической информации о предмете исследования за достаточно продолжительный период времени (статистики);
 \item при невозможности количественного моделирования всех факторов, оказывающих существенное влияние на принятие решения, т.\,е. когда в наличии имеется только информация, отражающей качественную сторону явлений; 
 \item в процессах, направление развития которых нетривиальным образом зависит от ещё не принятых решений;
 \item при исследовании качественно новых явлений и объектов, процесс развития которых уникален;
\end{itemize}

В подобных случаях важную роль для поддержки принятия решения играет суждение, высказанное человеком --- экспертом, специалистом --- которое, вообще говоря, не является объективным, и которое мы поэтому будем называть одним из следующих эквивалентных словосочетаний:
 \begin{itemize}
	\item субъективное суждение;
	\item экспертное мнение;
	\item ответ эксперта (на заданный ему вопрос). 
 \end{itemize}
 
 Эксперт может высказать суждение как самопроизвольно, так и в ответ на вопросы, специально сформулированные для поддержки принятия какого-либо решения. Процесс подготовки вопросов и прочих материалов, получения и последующего анализа экспертных мнений будем называть {\sl экспертным опросом} или {\sl экспертизой}.
 
 Экспертное мнение может быть выражено в виде развёрнутых рекомендаций и заключений, но этот случай не представляет интереса с математической точки зрения и в настоящей работе не рассматривается. Нас интересуют ситуации, когда можно выбрать некоторые числовые параметры, характеризующие важные для принятия решения аспекты предмета (или предметов) исследования, а эксперта (или экспертов) просят дать математическую оценку того или иного параметра. Ответ эксперта в этой ситуации логично назвать {\sl экспертной оценкой}, а всевозможные математические модели экспертных оценок будут рассмотрены ниже.  
 
 Хочется отметить, что иногда решение некоторой научно-практической проблемы уже назревает в мозгу специалистов, но ещё не оформлено в виде мыслей, имеющих достаточную чёткость для выражения. Экспертный опрос помогает осознать и формализовать эти мысли, после чего на суд лица, принимающего решение, выносятся не просто экспертные мнения по изначально предложенной экспертам схеме, а готовые варианты решения. 

 \subsection{Цель работы}
 Цель настоящей работы --- исследовать новый способ проведения экспертизы с применением высокоинформативных и при этом математически строгих экспертных оценок, которые моделируются методами теории возможностей Ю.~П.~Пытьева, попутно развить эти методы, поставить задачи, характерные для анализа экспертных оценок в области поддержки принятия решений, разработать эффективные алгоритмы и комплекс программ  для решения этих задач. 

