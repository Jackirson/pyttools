\todo[inline, color=orange]{Первая страница хорошая. Вторая лучше, третья~--- ещё лучше. Это касается и стилистики, и точности изложения.}

\subsection{Неопределённость и субъективные суждения}
\label{sec:intro_uncertainty}

Во всех сферах человеческой деятельности информация играет роль, важность которой трудно преувеличить. При этом, не всякая информация одинаково полезна: информация может быть неточной, неполной или вовсе ложной, и тогда её использование может повлечь за собой некоторые потери.  

Kzkz 

 В научной и деловой среде одним из признаков истинности информации является её объективность. При исследовании различных объектов, процессов и явлений предпочитают иметь дело с максимально объективной информацией о предметах исследования.  Объективной, например, считается информация, полученная в ходе наблюдения (в т.\,ч. с помощью измерительных приборов) за предметом исследования при фиксированных условиях и в результате анализа этих наблюдений, если точность полученного результата достаточно высока, а сам результат можно воспроизвести в повторном исследовании. 
  
 Такая информация далеко не всегда есть в наличии. 

% Поскольку информация есть мера определённости, то отсутствие той или иной информации можно назвать {\sl неопределённостью}. 
В контексте проводимого исследования отсутствие информации интересует нас с точки зрения неполноты знаний исследователя и заказчика исследования о предметах исследования. Неполноту знаний называют неизвестностью или неопределённостью~\ref{falomkina}.  Не вдаваясь в подробный философский анализ различия или сходства этих терминов, будем использовать их как синонимы.  Возьмём за данность, что для исследователя важны неизвестные ему свойства предметов исследования, но поскольку они неизвестны, приходится строить предположения относительно того, какими они могли бы быть. Например, при составлении прогноза погоды исследователю неизвестны многие свойства грозового фронта, особенно если рассматривать будущие состояния грозы, которые наступят через несколько часов или дней.

 С математической точки зрения, приходится строить модель неопределённости.  При этом следует различать стохастическую неопределённость (случайность) и нестохастическую неопределённость, часто обозначаемую термином <<нечёткость>>~\ref{falomkina}. 
 
 Теория вероятностей как математическая модель феномена случайности используется в теоретических и прикладных исследованиях благодаря двум её фундаментальным аспектам:
\begin{enumerate}
  \item математическому: теория вероятности базируется на теории меры и интеграла;
  \item эмпирическому: существуют простые, но математически обоснованные процедуры, позволяющие (при определённых условиях):
  \begin{itemize}
  \item  на основе серии наблюдений получить сколь угодно точную аппроксимацию вероятностной модели этих наблюдений;
  \item наоборот, при известной вероятностной модели предсказать событийно-частотные результаты наблюдений. 
  \end{itemize} 
\end{enumerate}

Эпитет <<стохастическая>> применительно к  неопределённости (неизвестности) результатов наблюдений за предметом исследования подразумевает, что существует вероятностная модель этих наблюдений  %, а сам предмет исследования является стохастическим объектом, т.\,е. моделируется некоторым вероятностным пространством 
$\OAPr$ в каждый момент времени. Здесь $\Alg$~--- алгебра событий над пространством элементарных событий $\Om$, а $\P(\cdot): \Alg \to \zo$~--- вероятностная мера. Чтобы построить вероятностную модель, нужно выбрать полную группу попарно несовместных элементарных событий, $\Om = \{\om_1, \om_2, \ldots\}$, описывающих все возможные результаты наблюдений, и приписать им значения вероятностной меры $\p_1 = \P(\{\om_1\}), \p_2 = \P(\{\om_2\}), \ldots$ из теоретических либо эмпирических соображений. 

Практический способ построения вероятностной модели существует не всегда. 

%Если этого сделать нельзя, будем говорить о нестохастической неопределённости. 
Например, в следующих случаях применение теории вероятностей не приводит к успеху:
\begin{enumerate}
  \item Пусть вероятностная модель отдельно взятого наблюдения существует, но вероятности элементарных событий~--- не константы; они могут  подчиняться своему собственному вероятностному закону, но могут и не подчиняться. Иными словами, пусть есть серия стохастических испытаний, моделируемых вероятностными пространствами $\OAPR{1}, \OAPR{2}, \ldots$, но вероятности элементарных исходов изменяются от испытания к испытанию, причём нет информации о зависимости $\Pr_{i}$ от индекса, и суммарное вероятностное пространство выглядит так: $\OAPR{1} \times \OAPR{2} \times \ldots$. Тогда большие объёмы данных наблюдений за реализацией событий $\om \in \Om$ оказываются неполными и противоречивыми, теоретически существующую вероятностную модель наблюдений построить нельзя, и неопределённость результатов наблюдений является \todo{Она \textbf{является} стохастической, но её вероятностная модель не может быть построена эмпирически}нестохастической. 
  \item Пусть испытание с неизвестным исходом нельзя многократно повторить при фиксированных условиях: можно сделать лишь несколько измерений, в предельном случае~--- лишь одно измерение. Тогда событийно-частотных (статистических) данных о предметах исследования не хватит для эмпирического восстановления вероятностной модели, даже если последняя теоретически существует. Неопределённость реализации событий $\om \in \Om$ в единичном испытании является \todo{То же замечание: она \textbf{является} стохастической, но её вероятностная модель не может быть построена эмпирически}нестохастической. 
  \item \todo{Вот это единственный пример действительно нестохастической неопределённости}Существуют явления, где присутствует неопределённость, но для которых построение вероятностной модели в принципе не имеет смысла. Широко используемые в настоящей работе {\sl субъективные суждения}~--- одно из таких явлений. 
\end{enumerate} 

Субъективное суждение о предметах исследования, в противоположность максимально объективной информации о них~--- это утверждение, высказанное человеком, которого  условно назовём {\sl экспертом}, намекая на то, что этот человек, скорее всего,~--- не произвольно выбранный человек, а специалист в тех предметных областях, где лежат предметы исследования, о которых он высказывается. Поэтому для словосочетания <<субъективное суждение>> в рамках настоящей работы используются также следующие синонимы: 
\begin{itemize}
	\item экспертное мнение;
	\item ответ эксперта (на заданный ему вопрос). 
 \end{itemize}
 
Неуверенность эксперта при ответе на какой-то заданный ему вопрос о предмете исследования порождает неопределённость, но она {\sl не тождественна} той неопределённости, которая ранее возникла из-за отсутствия объективной информации о предметах исследования, поскольку не тождественны объективная картина мира и картина мира с точки зрения эксперта. На экспертное мнение накладывается требование <<рациональности>>: будучи спрошен о чём-либо несколько раз подряд, эксперт выдаст один и тот же ответ. Поэтому неопределённость, которая может содержаться в ответе эксперта, является нестохастической даже в том случае, если для неопределённости результатов {\sl наблюдений} за предметами исследования можно построить вероятностную модель, хотя обычно к субъективным суждениям прибегают в случае, когда этого сделать нельзя. 

В настоящей работе рассматриваются задачи, где присутствуют все три перечисленных вида нестохастической неопределённости (далее просто <<неопределённость>>), тогда как стохастическая неопределённость остаётся, в целом, за рамками настоящей работы. Первые два типа неопределённости довольно близки и, в частности, относятся к исходной постановке задачи выбора объектов (выбор производится однократно, а если даже и нет, то условия выбора непредсказуемо изменяются). Третий тип неопределённости, касающийся субъективных суждений, связан с конкретным способом выбора объектов на основании модели условий выбора, создаваемой с помощью субъективных суждений. Методы моделирования субъективных суждений занимают центральное место в настоящей работе (см. \ref{sec:intro_asessment}, затем \todo{??}\ref{sec:math_methods_global}).  

\subsection{Принятие решений в условиях неопределённости}

Слово <<решение>> в русском языке имеет несколько оттенков: 
\begin{itemize}
  \item В математике решение задачи, в зависимости от её свойств, может быть найдено аналитически или численно, подобрано случайно или не совсем случайно, получено с использованием дополнительных соображений. Решение может быть или не быть единственным, а может и не существовать вовсе,  хотя во многих случаях можно переформулировать задачу так, чтобы решение всё же существовало и было единственным хотя бы с точностью до эквивалентных решений. 
  \item В более широком смысле, в практических задачах, решение часто требуется не только найти, но и {\sl принять} (иногда --- отвергнуть). Принятие такого <<волевого>> решения означает принятие человеком ответственности за  правильность решения. Человека, берущего на себя эту ответственность, будем называть {\sl лицом, принимающим решение}\footnote{Вообще говоря, лицом, принимающим решение, может быть как физическое, так и юридическое лицо, так и группа физических или юридических лиц.}. 
\end{itemize}
Например, задача выбора объектов в настоящей работе рассматривается и как проблема, требующая принятия <<волевого>> решения, и как математическая экстремальная задача. Решение экстремальной задачи называют оптимальным, и мы используем это слово в названии настоящей работы. Но следует помнить, что найденное оптимальное решение может быть отвергнуто лицом, принимающим <<волевое>> решение.

Существуют задачи, требующие принятия решения в условиях неопределённости (см. \ref{sec:intro_uncertainty}), когда нет времени или принципиальной возможности получить важные для принятия решения объективные данные о предметах исследования.
%Например, можно говорить о работе в условиях неопределённости, если верны одно или несколько из следующих утверждений:
\begin{itemize}
 \item отсутствуют фактические данные о предметах исследования за достаточно продолжительный период времени; 
 \item исследуется процесс, направление развития которого нетривиальным образом зависит от ещё не принятых третьими сторонами решений или событий, которые могут реализоваться или не реализоваться в будущем;
 \item исследуется качественно новое явление, процесс его развития уникален.
\end{itemize}

В условиях нехватки объективных данных важную роль для принятия решения играет субъективное суждение (см. \ref{sec:intro_uncertainty}), высказанное человеком, вообще говоря, без строгого обоснования и твёрдой опоры на факты. Этим человеком может быть и само лицо, принимающее решение, однако в некоторых случаях требуется помощь приглашённых экспертов. Например, это происходит, если предметы исследования не покрываются одной конкретной предметной областью, а находятся в самых разных предметных областях.  

Приглашённый эксперт может высказать мнение как в произвольной форме, так и в ответ на вопросы, специально сформулированные для поддержки принятия какого-либо решения. Процесс подготовки вопросов и прочих материалов, получения и последующего анализа экспертных мнений будем называть {\sl экспертным опросом} или {\sl экспертизой}. Лицо, принимающее решение, или субъект, от имени которого действует это лицо, выступает здесь в роли {\sl заказчика экспертизы}. От экспертов это лицо получает ответы на поставленные вопросы.
 
\subsubsection*{Замечание}
\begin{notice}
Если задача принятия решения в условиях неопределённости имеет научно-технический характер, то иногда её решение уже назревает в мозгу специалистов, работающих в соответствующей области. Однако, это решение может быть ещё не оформлено в виде мыслей, имеющих достаточную чёткость для выражения. Экспертный опрос, а именно, грамотно сформулированные аспекты предметов исследования и хорошие вопросы про них, помогает осознать и формализовать эти мысли, после чего на суд лица, принимающего решение, выносятся не просто экспертные мнения по изначально предложенной экспертам схеме, а готовые варианты решения. Похожие вещи происходят и в случае, когда лицо, принимающее решение~--- <<само себе эксперт>>: формализация процесса принятия решения сильно облегчает этот процесс. 
\end{notice}

\subsection{Экспертные оценки}
\label{sec:intro_asessment}

Экспертное мнение может быть выражено в виде развёрнутых рекомендаций и заключений, но этот случай не представляет интереса с математической точки зрения и в настоящей работе не рассматривается. Нас интересуют ситуации, когда можно выбрать некоторые числовые параметры $x_1, x_2, \ldots, x_n, n \in \N$, характеризующие важные для принятия решения аспекты предмета (или предметов) исследования, а эксперта (или экспертов) просят дать математическую оценку того или иного параметра. Ответ эксперта в этой ситуации логично назвать {\sl экспертной оценкой}. 

С математической точки зрения, для каждого исследуемого феномена следует построить математическую модель. Должна быть построена математическая модель ситуации, требующей принятия решения, и поставлена математическая задача поиска решения. Эта модель включает в себя, как составные части, модели предметов исследования вместе с выбранными параметрами. Например, пусть каждый из параметров задачи $x_i \in X_i$,  $i=1,\, \ldots,\, n$, лежит на числовом множестве, являющемся подмножеством действительной числовой оси: $X_i \subset \R$, $i=1,\, \ldots,\, n$. В частном случае, множества значений параметров могут совпадать, тогда $X = X_1 = X_2 = \ldots = X_n$. Множество значений параметра может быть, в частности, дискретным, например, $X = \{1, 2, ..., 10\}$ --- десятибалльная шкала значений параметра $x \in X$.

В свою очередь, существует математическая модель экспертной оценки параметра задачи как субъективного суждения. В роли экспертной оценки могут выступать различные математические объекты. Вот несколько вариантов моделей экспертной оценки, \todo[color=yellow]{Получается, что это не так. Чёткая оценка несёт больше всего информации, нечёткая~--- поменьше, интервал~--- ещё меньше}\todo[color=orange]{Речь идёт про информационную ёмкость {\sl модели}. А то, что Вы сказали,  относится к информативности разных случаев {\sl уже} в пределах модели нечёткой оценки.} в порядке возрастания количества информации, содержащейся в той или иной модели~\footnote{Речь идёт про информационную ёмкость разных моделей экспертной оценки, измеряемой, например, в байтах. Если же сравнивать различные ситуации в пределах модели нечёткой экспертной оценки (пункт~3), то интуитивные представления об <<информативности>> соответствуют следующему порядку: <<чёткая>> оценка несёт больше всего информации, нечёткая~--- поменьше, интервал, он же носитель нечёткой оценки~--- ещё меньше. Это утверждение поясняется в разделе \todo{??} о линейном псевдопорядке на распределениях нечётких элементов.}: 
\begin{enumerate}
  \item Одно число из того же подмножества (действительных либо натуральных чисел) $X$: $\hat{x} \in X$. Количество информация такого объекта $I = I_w$, где $I_w$~--- количество информации для представления на ЭВМ одного числа из $X$ (например, в байтах);
  \item Пара действительных чисел $x^{(1)}, x^{(2)} \in X$, задающая интервал $[x^{(1)}, x^{(2)}]$. Количество информации $I = 2I_w$;
  \item Заданная таблицей, графиком или иным образом функция, определённая на заданном интервале $[x^{(1)}, x^{(2)}]$, принимающая значения на отрезке $\zo$, см. рисунок \todo{??}\ref{ris:expert_fuzzy_general}. Такая модель оценки позволяет эксперту указать не только интервал $[x^{(1)}, x^{(2)}]$, но и некоторые веса отдельных значений внутри интервала. Эти веса могут интерпретироваться, например, как степень уверенности эксперта в каждом из этих значений. Если в дискретном представлении для ЭВМ $X = \{1, 2, ..., x_{k}\}$, то количество информации $I = kI_w$. 
\end{itemize}
%Увеличенное, по сравнению с предыдущими вариантами, количество информации в этой модели экспертной оценки, может сильно помочь лицу, принимающему решение, как будет показано ниже. 
%Оценки такого вида и с такой интерпретацией значений на отрезке $\zo$ мы будем, в противоположность <<чётким>> оценкам из 1-го пункта, называть {\sl нечёткими}.

Будем считать, что эксперт выставляет оценку осознанно (не <<наобум>>) и заинтересован в достижении высокого качества экспертизы (обладает ответственностью). %(как его в этом заинтересовать?) 
Поэтому наложим на экспертные оценки следующие условия <<рациональности>>:
\begin{enumerate}
 \item будучи спрошен несколько раз подряд об одном и том же, эксперт выдаст один и тот же ответ; 
 \item эксперт выразит своё мнение максимально полно в рамках используемой математической модели экспертной оценки.
\end{enumerate}

Как уже говорилось, субъективные суждения~--- объекты не стохастической природы. Экпертные оценки~--- не оценки из области математической статистики. Эксперт не является <<измерительными прибором>> с присущей последнему погрешностью измерений. Поэтому, в отличие от модели формирования данных измерений, модель формирования экспертной оценки не является стохастической, и экспертная оценка не является случайной в теоретико-вероятностном смысле. Модель экспертной оценки, в отличие от модели данных измерений, не требует процедуры эмпирического восстановления. Модель экспертной оценки, сколь бы сложной она не была, сразу и целиком задаётся экспертом, например, при ответе на поставленный ему вопрос.
 
Субъективность суждения эксперта не подразумевает запрета опираться в том числе и на объективную информацию, если она всё-таки имеется в наличии в том или ином объёме. Но в математической модели неопределённости, порождаемой субъективными суждениями, мы не будем учитывать такую, в общем случае, отсутствующую информацию и считаем, что эксперт <<извлекает>> экспертную оценку непосредственно из своего сознания. 

Для анализа экспертных оценок надо определить допустимые математические операции над ними и необходимые для этой цели дополнительные свойства используемых математических объектов, что и делается в разделе \todo[color=orange]{Пожалуйста, не обращайте внимание на, возможно, неработающие внутренние ссылки. Я их просто делаю заранее, а соотв. разделы на очереди.}\ref{sec:math_methods_global}. В подразделе \ref{sec:math_methods_ours} рассказывается про теорию возможностей Ю.~П.~Пытьева~\cite{possbook}, в рамках которой строится математическая модель экспертной оценки, используемая в настоящей работе.  

%Экспертная оценка есть по определению субъективное мнение эксперта. Так называемая субъективность суждения эксперта не подразумевает запрета опираться в том числе и на объективную информацию, если она всё-таки имеется в наличии в том или ином объёме. Но в математической модели экспертной оценки мы не будем учитывать такую, в общем случае, отсутствующую информацию и считаем, что эксперт извлекает оценку непосредственно из своего сознания. 


\subsection{Цель работы}
Это пока отложим. Цель лучше вписывать в текст в самом конце работы, когда её формулировка приобретает максимальную конкретность.
%Цель настоящей работы --- исследовать новый подход к проведению экспертизы, используя высокоинформативные, но при этом математически строгие модели экспертных оценок в рамках теории возможностей Ю.~П.~Пытьева. Новый подход будет продемонстрирован на примере нескольких задач из области поддержки принятия решений с использованием экспертных оценок. Для решения этих задач на ЭВМ будет разработан демонстративный комплекс программ.
%Для решения этих задач будет раз разработать эффективные алгоритмы и комплекс программ для решения этих задач. 

