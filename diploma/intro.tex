
\subsection{Принятие решений в условиях неопределённости}
\label{sec:basic_intro}

Слово <<решение>> в русском языке имеет несколько оттенков. В математике чаще всего требуется строгая постановка задачи, после чего решение задачи, в зависимости от её свойств, может быть найдено аналитически или численно, случайно или не совсем случайно подобрано, получено с использованием дополнительных соображений,  априорной информации и с помощью анализа фактов. Решение может быть единственным, а может не быть, или не существовать вовсе. В двух последних случаях часто можно переформулировать задачу так, чтобы решение всё же существовало и было единственным хотя бы с точностью до эквивалентных решений. 

\todo{Плохо, что в этом абзаце перемешиваются решения в контексте мат. задач и в контексте человеческих <<волевых>> решений. Эти контексты нужно чётко разграничить, например, 1-ый абзац~--- первый контекст, далее~--- второй}В более широком смысле, в практических и прикладных задачах, решение часто требуется не только найти, но и {\sl принять} (иногда --- отвергнуть). Принятие решения означает принятие человеком ответственности за правильность решения, будь оно решением математической задачи или нет. Человека, берущего на себя эту ответственность, будем называть {\sl лицом, принимающим решение}, а действия и средства, помогающие ему в этом\todo{Используй везде ``${\sim---}$'', чтобы тире не отрывалось от предшествующего текста}~---  поддержкой принятия решения. Решение экстремальной задачи в математике называют оптимальным, и мы используем это слово в названии настоящей работы. Но следует помнить, что найденное оптимальное решение может быть неправильным в более широком, практическом смысле --- с точки зрения лица, принимающего решение. 

Во всех сферах человеческой деятельности информация играет роль, важность которой трудно преувеличить. В научной и деловой среде часто предпочитают максимально объективную информацию о предметах исследования, в качестве которых выступают некие объекты, процессы и явления. Например, объективной считается информация, полученная с помощью измерительных приборов и в результате анализа уже свершившихся фактов, если точность полученного результата достаточно высока, а сам результат можно воспроизвести в повторном исследовании. 

На основе объективной информации часто можно сразу рассчитать оптимальное решение или критерий принятия решения. Но такая информация есть в наличии не всегда. Существуют задачи, требующие принятия решения в условиях неопределённости --- когда одних лишь объективных данных о предмете исследования не хватает для поиска решения, оптимального в некотором смысле, и нет времени или даже принципиальной возможности их  получить. Например, можно говорить о работе в условиях неопределённости, если верны некоторые из следующих утверждений:
\begin{itemize}
 \item отсутствует фактическая информации о предмете исследования за достаточно продолжительный период времени (статистические данные);
 \item нет возможности количественного моделирования всех факторов, оказывающих существенное влияние на принятие решения, в наличии имеется только информация, отражающей качественную сторону явлений; 
 \item исследуется процесс, направление развития которого нетривиальным образом зависит от ещё не случившихся событий, которые могут случиться или не случиться в будущем;
 \item исследуется качественно новое явление или объект в процессе их развития, который уникален.
\end{itemize}

Та или иная процедура принятия решения может считаться оптимальной в практическом смысле. Решение может приниматься единолично. Но во многих случаях лицо, принимающее решение, приглашает {\sl экспертов}, компетентных специалистов. Например, это происходит,  если предметы исследования не покрываются одной конкретной предметной областью, а находятся в самых разных предметных областях. 
% к этому абзацу нужны ссылки на другие примеры стратегий выбора в литературе (Миша?)\

В условиях нехватки объективных данных важную роль для принятия решения играет суждение, высказанное экспертом, человеком, без строгого обоснования и твёрдой опоры на факты. Такое суждение, вообще говоря, не является объективным, поэтому будем обозначать его одним из следующих эквивалентных словосочетаний:
 \begin{itemize}
	\item субъективное суждение;
	\item субъективное мнение;
	\item экспертное мнение;
	\item ответ эксперта (на заданный ему вопрос). 
 \end{itemize}

Приглашённый эксперт может высказать суждение как \todo{В произвольной форме}самопроизвольно, так и в ответ на вопросы, специально сформулированные для поддержки принятия какого-либо решения. Процесс подготовки вопросов и прочих материалов, получения и последующего анализа экспертных мнений будем называть {\sl экспертным опросом} или {\sl экспертизой}. Лицо, принимающее решение, или субъект, от имени которого действует это лицо, выступает здесь в роли {\sl заказчика экспертизы}. От экспертов оно получает ответы на поставленные вопросы.
 
Если задача принятия решения в условиях неопределённости имеет научно-технический характер, то иногда её решение уже назревает в мозгу специалистов, работающих в соответствующей области. Однако, это решение может быть ещё не оформлено в виде мыслей, имеющих достаточную чёткость для выражения. Экспертный опрос помогает осознать и формализовать эти мысли, после чего на суд лица, принимающего решение, выносятся не просто экспертные мнения по изначально предложенной экспертам схеме, а готовые варианты решения. 

\subsection{Экспертные оценки}

Экспертное мнение может быть выражено в виде развёрнутых рекомендаций и заключений, но этот случай не представляет интереса с математической точки зрения и в настоящей работе не рассматривается. Нас интересуют ситуации, когда можно выбрать некоторые числовые параметры $x_1, x_2, \ldots, x_n, n \in \N$, характеризующие важные для принятия решения аспекты предмета (или предметов) исследования, а эксперта (или экспертов) просят дать математическую оценку того или иного параметра. Ответ эксперта в этой ситуации логично назвать {\sl экспертной оценкой}. 

С математической точки зрения, для каждого исследуемого феномена следует построить математическую модель. Должна быть построена математическая модель ситуации, требующей принятия решения, и поставлена математическая задача поиска решения --- \todo{Точность и оптимальность~--- совершенно разные характеристики решения. Некорректно ставить вопрос о выборе только одной из них}точного либо оптимального. Эта модель включает в себя, как составные части, модели предметов исследования вместе с выбранными параметрами. Например, пусть каждый из параметров задачи $x_i \in X_i$ лежит на числовом множестве, являющемся подмножеством действительной числовой оси: $X_i \subset \R,\ i=1,\, \ldots,\, n$. В частном случае, множества значений параметров могут совпадать, тогда $X = X_1 = X_2 = \ldots = X_n$. Множество значений параметра может быть и дискретным, например, $X = \{1, 2, ..., 10\}$ --- десятибалльная шкала значений параметра $x \in X$.

В свою очередь, существует математическая модель экспертной оценки параметра задачи. %Экспертную оценку параметра $x \in X \subset R$ мы обозначим $\hat{x}$, понимая под этим 
В роли экспертной оценки может выступать любой выбранный заказчиком экспертизы математический объект: 
\begin{itemize}
  \item одно действительное число из того же множества $X$: $\hat{x} \in X$;
  \item пара действительных чисел $x^{(1)}, x^{(2)} \in X$, задающая интервал $[x^{(1)}, x^{(2)}]$;
  \item \todo{Совсем не понятно. Что за функция? Какими св-вами обладает?}заданная таблицей, графиком или иным образом функция, определённой на специальном пространстве и обладающей специальными свойствами.
\end{itemize}
Для анализа экспертных оценок надо в каждом случае определить допустимые математические операции над ними.

\todo{Очень сбивчивый абзац. Может его пока убрать?}Экспертная оценка --- это не оценка из области математической статистики. Выше речь шла про условия неопределённости, когда объективная информация о предметах исследования, а значит, и о параметрах задачи, вообще говоря, отсутствует. В теории вероятностей, которую называют теорией для моделирования \todo{В теории возможностей неопределённость тоже определённая}<<определённой неопределённости>>, невозможность наблюдения за системой с целью сбора статистических данных означает невозможность сконструировать рабочую вероятностную модель. Вероятностная (синоним --- стохастическая) модель должна \todo{Не обязательно. В случае с монеткой или игральной костью, или в стат. физике вероятн. модель строится теоретически}<<подпитываться>> данными для того, чтобы присвоить элементарным событиям вероятностного пространства определённые значения вероятностной меры. Значения вероятности должны иметь вполне конкретную частотную интерпретацию. Такой процесс <<подпитки>> эмпирическими данными называют эмпирическим восстановлением вероятностной модели.

В нашем же понимании феномена неопределённости, математическим определением условий неопределённости в  является следующее утверждение. 
%\label{box:ambiguity}
\begin{center} \todo{А сто\'{и}т ли за ними вероятностная модель? Наше незнание совсем не обязательно связано со стохастической природой объекта исследования}\fbox{ 
\begin{minipage}{0.9 \textwidth}
 Истинные значения параметров задачи, вообще говоря, неизвестны на момент принятия решения. Для эмпирического восстановления вероятностной модели этих параметров не хватает данных. 
\end{minipage}
} \end{center} 

Экспертная оценка есть по определению субъективное мнение эксперта. Так называемая субъективность суждения эксперта не подразумевает запрета опираться в том числе и на объективную информацию, если она всё-таки имеется в наличии в том или ином объёме. Но в математической модели экспертной оценки мы не будем учитывать такую, в общем случае, отсутствующую информацию и считаем, что эксперт извлекает оценку непосредственно из своего сознания. При этом будем считать, что эксперт выставляет оценку осознанно (не <<наобум>>) и заинтересован в достижении высокого качества экспертизы. %(как его в этом заинтересовать?) 
В этом случае:
\begin{enumerate}
 \item эксперт выразит своё мнение максимально полно в рамках математическом модели экспертной оценки;
 \item будучи спрошен несколько раз подряд об одном и том же, эксперт выдаст один и тот же ответ. 
\end{enumerate}

\todo{Вот это тоже не понятно. Что за <<информативность выбранной мат. модели экп. оценки>>?}Первый пункт списка означает, что количество информации, извлечённой из сознания эксперта в процессе экспертизы, равна минимальной из двух величин: количество информации, доступной эксперту сознательно и подсознательно при ответе на заданный ему вопрос, и информационная <<ёмкость>>, или информативность, выбранной математической модели экспертной оценки.

Второй пункт списка означает следующее. Эксперт не является <<измерительными прибором>> с присущей последнему погрешностью измерений. Поэтому, в отличие от модели формирования данных измерений, модель формирования экспертной оценки не является стохастической, и экспертная оценка не является случайной в теоретико-вероятностном смысле. Посредством многократного опроса эксперта нельзя восстановить вероятностную модель параметра задачи. Более того, использование одних лишь экспертных оценок с учётом сформулированного выше определения условий неопределённости делает использование стохастических моделей вообще не приемлемым в настоящей работе. Стоит подчеркнуть, что модель экспертной оценки, в отличие от модели данных измерений, не требует процедуры эмпирического восстановления. Модель экспертной оценки, сколь бы сложной она не была, сразу и целиком задаётся экспертом при ответе на поставленный ему вопрос.

Рассуждения об информативности, преимуществах и недостатках различных математических моделей экспертной оценки продолжаются в обзоре математических методов моделирования и анализа экспертных оценок. Там же рассказывается про теорию возможностей Ю.~П.~Пытьева~\cite{possbook}, % пробная ссылка на Пытьева
в рамках которой строится математическая модель экспертной оценки, используемая в настоящей работе.  

\subsection{Цель работы}
\todo[color=orange]{Это пока отложим. Цель лучше вписывать в текст в самом конце работы, когда её формулировка приобретает максимальную конкретность}Цель настоящей работы --- исследовать новый подход к проведению экспертизы, используя высокоинформативные, но при этом математически строгие модели экспертных оценок в рамках теории возможностей Ю.~П.~Пытьева. Новый подход будет продемонстрирован на примере нескольких задач из области поддержки принятия решений с использованием экспертных оценок. Для решения этих задач на ЭВМ будет разработан демонстративный комплекс программ.
%Для решения этих задач будет раз разработать эффективные алгоритмы и комплекс программ для решения этих задач. 

