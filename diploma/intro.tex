
\subsection{Субъективная информация и экспертные оценки}

Во многих сферах человеческой деятельности, включая экономические, социологические и другие научные исследования, немаловажную роль играет информация, полученная не с помощью измерительных приборов, а высказанная людьми --- экспертами, специалистами в своем деле. Такую информацию, в противоложность максимально объективным данным измерений, будем называть {\sl субъективной}. 

Использование субъективной информации помогает исследовать сложные объекты в условиях неопределенности, которая может возникнуть:
\begin{itemize}
 \item при отсутствии фактической информации за достаточно продолжительный период времени (статистики);
 \item при наличии информации, отражающей только качественную сторону явлений, и невозможности количественного моделирования всех факторов, оказывающих существенное влияние на принимаемое решение;
 \item в процессах, направление развития которых нетривиальным образом зависит от еще не принятых решений;
 \item при исследовании качественно новых процессов и явлений, т.\,е. в ситуациях, когда процесс развития объекта прогнозирования уникален.
\end{itemize}

Субъективная информация формируется в ответ на некий вопрос к эксперту. Отметим, что такой вопрос обычно рассчитан на получение от эксперта не только сознательной, но и подсознательной информации. Часто бывает так, что решение некоторой научно-практической проблемы уже существует в мозгу специалистов, но ещё не оформлено в виде мыслей, имеющих достаточную чёткость для выражения и воспроизведения. В этом случае опрос группы экспертов может спровоцировать формирование таких мыслей.

Ответ эксперта на заданный ему вопрос может быть выражен в виде развернутых рекомендаций или заключений, но этот случай не представляет интереса с математической точки зрения и в настоящей работе не рассматривается. Мы рассмотрим ситуации, когда эксперта просят дать математическую оценку того или иного параметра (аспекта) объекта исследования. Ответ эксперта в этой ситуации логично назвать {\sl экспертной оценкой}. 

Будем считать, что эксперт выставляет оценку осознанно и заинтересован в достижении высокого качества научного исследования. В этом случае:
\begin{itemize}
 \item эксперт выразит своё мнение максимально полно, используя все предоставленные ему средства, о которых будет сказано в дальнейшем изложении;
 \item будучи спрошен несколько раз подряд одно и то же, эксперт выдаст один и тот же ответ. 
\end{itemize}

Эксперт не является <<измерительными прибором>> с присущей последнему погрешностью измерений. Поэтому, в отличие от модели формирования данных измерений, модель формирования экспертной оценки не является стохастической, т.\,е. экспертная оценка не является случайной в теоретико-вероятностном смысле. В связи с этим стоит подчернкуть, что модель экспертной оценки, в отличие от модели данных измерений, не требует процедуры эмпирического восстановления. Модель экспертной оценки, сколь бы сложной она не была, сразу и целиком задаётся экспертом при ответе на поставленный ему вопрос.

\subsection{Традиционные экспертные оценки}

Самой простой и широко распространенной моделью экспертной оценки является одно число: $x \in X$, где, например, $X = \R \defeq (-\infty, \infty)$ (оценка вещественного параметра), или $X = \{1, 2, ..., 10\}$ (оценка по десятибалльной шкале). Ограничения такой модели: в её рамках эксперт не может выразить неуверенность, частичное или полное незнание, возможность нескольких сценариев развития исследуемого объекта и т.\,д.

Следующей по сложности моделью экспертной оценки является интервал $[x_{min},\,x_{max}], {x_{min} \in~X}, {x_{max} \in~X}, {x_{min} \leq x_{max}}$. Такие <<интервальные>> оценки нельзя путать с интервальными оценками в математической статистике, где интервал со случайными границами покрывает оцениваемую случайную величину с заданной вероятностью. <<Интервальные>> экспертные оценки следует рассматривать как заданные экспертом не случайные числа, между которыми, по его мнению, заключён оцениваемый параметер $\tilde x \in X$, не известный эксперту на момент выставления оценки. 

Аналогично арифметическим действиям, которые можно выполнять с одиночнымии числами, можно ввести арифметические действия, а также операции математического анализа, над числовыми интервалами. Это делается в рамках интервальной математики \cite{1,2}. С интервалами $A = [a_1,\,a_2]$ и $B = [b_1,\,b_2]$ можно выполнять, например, следующие действия:
\begin{equation*}
\begin{split}
 A + B &= [a_1+b_1,\,a_2+b_2];\\
 A - B &= [a_1-b_2,\,a_2-b_1];\\
 A \mult B &= [\inf\{a_1b_1,a_1b_2,a_2b_1,a_2b_2\},\,\sup\{a_1b_1,a_1b_2,a_2b_1,a_2b_2\}]
\end{split} 
\end{equation*}

% TODO: написать про доверительные интервалы.

\subsection{Нечёткие экспертные оценки}
 
В случае использования <<интервальных>> оценок эксперт не указывает предпочтительность тех или иных значений $x \in [x_{min},\,x_{max}]$. Если же она указана, получается оценка нового типа --- {\sl нечёткая} экспертная оценка. Эти более сложные оценки могут моделироваться разными способами, из которых мы рассмотрим два --- теорию возможностей (т.\,в.) Л.~А.~Заде \cite{3} и т.\,в. Ю.~П.~Пытьева \cite{4}. Но сначала введём общие понятия.

Будем говорить, что задана нечёткая экспертная оценка, если для всех $x \in X$ экспертом задана величина $\p_{\tilde x}(x) \in \zo$, выражающая относительную возможность\footnote{Хотя в быту в таком контексте часто говорят <<вероятность>>, во всех математических теориях и в настоящей работе важно подчеркнуть нестохастический характер субъективной информации, см. Введение. В некоторых работах, например, \cite{4_1}, используется термин <<субъективная вероятность>>.} того, что оцениваемый параметр $\tilde x$ примет значение $x$.
Иными словами, задана функция $\p_{\tilde x}(\cdot): X \rightarrow \zo$, называемая функцией распределения возможностей значений $x$ параметра $\tilde x$. При этом $\p_{\tilde x}(x) = 1$ соотвествует наиболее возможным $x$, а $\p_{\tilde x}(x) = 0$ --- принципиально невозможным, по мнению эксперта, значениям $\tilde x$. Пример нечёткой оценки приведён на рис. \ref{ris:fuzzy_ass00}\footnote{На рис. \ref{ris:fuzzy_ass00} функция $\p_{\tilde x}(\cdot)$ изображена непрерывной. Вообще говоря, задання экспертом функция $\p'_{\tilde x}(\cdot)$ не обязана быть непрерывной. В т.\,в. Заде $\p_{\tilde x}(\cdot)$ считается непрерывной, подразумевая, что $\p'_{\tilde x}(\cdot)$ дополняется по непрерывности до $\p_{\tilde x}(\cdot)$. В т.\,в. Пытьева $\p_{\tilde x}(\cdot)$ считается измеримой, но не обязана быть непрерывной.}.

\begin{figure}[h]
\center{\includegraphics[width=0.7\linewidth]{fuzzy_ass00}}
\caption{\small Пример нечёткой оценки. По горизонтальной оси отложены значения $x$ оцениваемого параметра $\tilde x$. По вертикальной оси отложена возможность того, что $\tilde x = x$. Наиболее возможным эксперт признал значение $6$, менее возможными ---  $\{4,5,7,8\}$, и совсем невозможными --- $\{1,2,3,9,10\}$.}
\label{ris:fuzzy_ass00}
\end{figure}

Нечёткая оценка позволяет эксперту выразить как точное знание $\tilde x$, положив $\tilde x = x_0, x_0 \in X$ единственно возможным (рис. \ref{ris:fuzzy_ass01}, слева), так и полное незнание $\tilde x$, положив все $x \in X$ равновозможными (рис. \ref{ris:fuzzy_ass01}, справа), а также любой промежуточный случай.

\begin{figure}[h]
\center{\includegraphics[width=0.35\linewidth]{fuzzy_ass01a}
        \includegraphics[width=0.35\linewidth]{fuzzy_ass01b} }
\caption{\small Нечёткие оценки, выражающие точное знание (слева) и полное незнание (справа). }
\label{ris:fuzzy_ass01}
\end{figure}

\subsection{Теория возможностей Л.~А.~Заде как средство моделирования нечётких экспертных оценок}

Т.\,в. Заде основана на теории нечётких множеств \cite{5,6}. Связь между этими теориями следующая. Пусть нечёткое множество $F \subseteq X$ с функцией принадлежности $\mu_{F}(\cdot): X \rightarrow \zo$ является нечётким ограничением \cite{3} оцениваемого параметра $\tilde x$. Например, $\tilde x$ --- возраст человека, $F$ --- утверждение <<этот человек --- молодой>>, а $\mu_{F}(x)$ интерпретируется как <<степень соответствия>> возраста $x$ утверждению $F$. Тогда по определению полагается $\p_{\tilde x}(\cdot) \defeq \mu_{F}(\cdot)$. В нашем примере это означает, что возможность того, что возраст человека окажется равным $x$, если эксперт утверждает, что <<человек молодой>>, численно равна <<степени соответствия>> этого возраста нечёткому множеству <<молодость>>. 

В теории нечётких множеств вводятся операции над нечёткими множествами $A, B$ в терминах их функций принадлежности: 
\begin{equation*}
  \mu_{A \cup B}(\cdot) \defeq \sup\{\mu_{A}(\cdot),\mu_{B}(\cdot)\}; \ \ \mu_{A \cap B}(\cdot) \defeq \inf\{\mu_{A}(\cdot),\mu_{B}(\cdot)\}; 
\end{equation*}
Аналогично арифметическим действиям, которые можно выполнять с одиночнымии числами, вводятся бинарные операции $f(A, B)$, где $f(\cdot, \cdot): \R \times \R \rightarrow \R$ --- любая числовая функция, например, $\plus$ или $\mult$ \cite{7}: 
\begin{equation*}
  \mu_{f(A, B)}(z) = \underset{(x, y): f(x, y) = z}\sup \inf\{\mu_{A}(x),\mu_{B}(y)\}, \ x,y,z \in X; 
\end{equation*}
Поскольку $\p_{\tilde x}(\cdot) \defeq \mu_{F}(x)$, то и для функций $\p_{\tilde x}(\cdot)$, т.\,е. для нечётких экспертных оценок, в т.\,в. Заде используются арифметические действия. Более того, и это принципиально важно: содержательно интерпретируются все конкретные значения $\p_{\tilde x}(\cdot)$, например, $0.5, 0.2$. Такой подход имеет следующий недостаток.

Когда эксперт выставляет оценку в виде балла, например, $X = \{1, 2, ..., 10\}$, существует специальный регламент относительно того, какой смысл несёт каждый балл $x \in X$. Каждый случай проведения экспертизы требует разработки регламента и установления соглашения между экспертами. Этот процесс постоянно совершенствуется, но до сих пор в таких регламентах остается немало неопределенности. Представим теперь, что необходимо договориться не только по поводу смысла значений $x \in X$, но и по поводу смысла значений $\p_{\tilde x}(\cdot) \in \zo$. Это создаст неподъемную нагрузку на экспертов. 

Этот недостаток устранён в рамках т.\,в. Пытьева.

\subsection{Теория возможностей Ю.~П.~Пытьева как средство моделирования нечётких экспертных оценок}

Т.\,в. Ю.~П.~Пытьева построена аксиоматически, аналогично построению теории вероятностей \cite{4,8}. Пусть имеется множество элементарных исходов $\Om$ и задана $\sigma$-алгебра событий $\Alg \subseteq 2^{\Om}$. Возможностью называется всякая измеримая функция $\P(\cdot): \Alg \rightarrow \zo$, обладающая определенными свойствами, которые будут определены ниже. Тройка $\OAP$ называется пространством с возможностью. 

Определим шкалу $\scL$ значений возможности $\P(\cdot)$ как интервал $\zo$ с естественной упорядоченностью $\leqslant$ и двумя бинарными операциями, $\plus: \zo \times \zo \rightarrow \zo$ и $\mult: \zo \times \zo \rightarrow \zo$. С точки зрения эксперта, $\scL$ --- это правила, по которым он размещает возможности событий из $\Alg$ на отрезке $\zo$. 

Пусть $\Gamma$ --- группа строго монотонно возрастающих непрерывных $\gamma(\cdot): \zo \rightarrow \zo,\ \gamma(0)=0,\ \gamma(1)=1$ с групповой операцией $\circ: f \circ g = f(g(\cdot))$. Возможности $\P$ и $\P'$ назовем {\sl эквивалентными}, если $\P' = \gamma(\P)$. Эквивалентные возможности образуют класс возможностей $\PP$ и соотвествующий класс пространств с возможностью \{$\OAP\ |\ \P \in \PP $\}. Принцип относительности (см. \cite{8}) гласит, что утверждения и модели, полученные с использованием возможности, содержательны в том и только в том случае, если они остаются верными при замене данной возможности $\P$ на любую эквивалентную ей $\P'$. 
%Иными словами, не важен выбор конкретной шкалы $\scL$, потому что группа $\Gamma$ порождает группу $\tilde \Gamma$ автоморфизмов $\scL$.

Из принципа относительности следует, что операции $\plus$ и $\mult$ над значениями возможностей двух произвольных множеств $A, B \in \Alg$, т.е. числами $a = \P(A) \in \zo, b = \P(B) \in \zo$, должны удовлетворять условиям:
\begin{center}
  $\gamma(a \plus b) = \gamma(a) \plus \gamma(b),\ $
  $\gamma(a \mult b) = \gamma(a) \mult \gamma(b) $
\end{center}
Также эти операции должны быть коммутативны и не выводить за пределы $\zo$:
\begin{center}
  $a \plus b = b \plus a,\ a \plus 1 = 1,\ a \plus 0 = a, $\\
  $a \mult b = b \mult a,\ a \mult 1 = a,\ a \mult 0 = 0, $
\end{center}
В \cite{8} показано, что единственный способ получить все эти свойства в классе непрерывных операций --- определить
\begin{equation*}
 a \plus b \defeq \sup\{a,b\}; \ \  a \mult b \defeq \inf\{a,b\} 
\end{equation*}
Эти операции естественным образом распространяются на конечное или счётное число операндов. 

Таким образом, возможность --- мера, принимающая значения на шкале $\scL = \scale$, где сложение понимается как точная верхняя грань, а умножение --- как точная нижная грань. По аналогии с вероятностной мерой постулируются два свойства возможности:
\begin{enumerate}
 \item аддитивность: $\P(A \cup B) = \P(A) \plus \P(B)\ \forall A,B \in \Alg$, где $a \plus: b \defeq \sup\{a,b\}, a, b \in \zo$;\footnote{Из-за переопределенной операции сложения в свойстве (1), в отличие от аналогичного свойства вероятностей, не требуется $A \cup B = \o{}$.}
 \item нормировка на единицу: $\P(\Om) = \underset{\omega \in \Om}\sup\ \P({\omega}) = 1$;
\end{enumerate}

Моделью оцениваемого параметра в т.\,в. Пытьева выступает математический объект, аналогичный понятию случайной величины в теории вероятностей --- нечёткий элемент $\tilde x(\cdot): \Om \rightarrow X$. Моделью нечёткой экспертной оценки служит функция распределения возможностей нечёткого элемента $\p_{\tilde x}(\cdot) \defeq \P(\{\tilde x(\omega) = x\})$. 

Итак, принципиальное отличие т.\,в. Пытьева от т.\,в. Заде в том, что в ней содержательно интерпретиуются не все конкретные значения $\p_{\tilde x}(\cdot)$, а только {\sl упорядоченность} этих значений, и отдельно --- неподвижная относительно группы $\Gamma$ точка <<0>>, так что $\p_{\tilde x}(x_0) = 0$ означает принципиальную невозможность значения $x_0 \in X$. Используя такую модель экспертной оценки, можно предоставить эксперту свободу выбора своей субъективной шкалы $\scL$, и экспертные оценки на рис. \ref{ris:fuzzy_ass03} будут выражать одно и то же экспертное мнение.

\begin{figure}[h]
\center{\includegraphics[width=0.7\linewidth]{fuzzy_ass03}}
\caption{\small Пример двух эквивалентных экспертных оценок.}
\label{ris:fuzzy_ass03}
\end{figure}

Далее в настоящей работе рассмотрены задачи, возникающие при использовании нечётких оценок с позиции т.\,в. Пытьева. Приведены алгоритмы решения этих задач, которые благодаря эффективности переопределенных операций сложения и умножения шклаы $\scL$ оказываются P-эффективными.