
В настоящей работе рассмотрены задачи, связанные с  использованием экспертных оценок в форме распределений возможности в рамках теории возможностей Ю.~П.~Пытьева (такие оценки были названы нечёткими). В рамках упомянутой теории были впервые получены следующие результаты:
\begin{enumerate}
\item
Поставлена и решена задача принятия решения о выборе наиболее качественных объектов на основе нечётких экспертных данных об их характеристиках, сформулированная как задача на минимум возможности принять неверное решение (ошибиться);
\item
Поставлена и решена задача нахождения коллективного мнения экспертов, не противоречащего мнениям отдельных экспертов, сформулированная как задача нахождения верхней грани на множестве возможностных распределений (экспертных оценок), заданных экспертами, используя введённое на множестве распределений отношение квазипорядка.
\item 
%\todo[inline]{Сократи этот пункт, не надо ничего лишнего. Например: разработан численный метод ..., позволивший решать данную задачу за время порядка O(...) вместо O(...) при использовании метода полного перебора, где $n$~--- ..., $m$~--- ...}
Разработан численный метод решения задачи о выборе наиболее качественных объектов, %который оказывается вычислительно эффективным благодаря эффективности операций <<$\sup$>> и <<$\inf$>> в качестве сложения и умножения значений возможности соответственно. 
позволяющий решить эту задачу за время порядка $O(nm)$ вместо $O\big(\abs{X}^{nm}\big)$ в случае полного перебора всех элементраных событий, где $n$ --- количество объектов, $m$ --- количество параметров у каждого объекта, а $\abs{X}$ --- размер множества $X$ значений параметров.
%В задаче имеется, возможно, большое число  $n$ оцениваемых объектов, каждый из которых имеет $m$ параметров, из которых складывается качество $x \in X$ того или иного объекта. Предложенный алгоритм позволяет в случае конечного $X$ вычислить значение возможности ошибки за $O(nm)$ итераций вместо $O(\abs{X}^{nm})$ в случае полного перебора всех элементраных событий (тут $\abs{X}$ есть размер множества $X$).
 \item
Предложен эвристический численный метод решения задачи нахождения коллективного мнения экспертов, позволяющий решить эту задачу за время порядка $O\big(R(nm)^2\big)$ вместо $O\big(R\abs{X}^{2nm}\big)$ в случае построения совместного распределения возможностей всех параметров всех объектов, где $n,m,\abs{X}$~--- те же, что выше, а $R$ --- количество экспертов. %оказывающийся вычислительно эффективным за счёт отказа от вычисления точной верхней грани в пользу <<не слишком сильно>> отличающейся от неё верхней грани. Для тех же, возможно, больших чисел $n$, $m$ %
%Вычисления проводятся не с совместнымым распределениями возможностей всех параметров всех объектов, что имело бы сложность $O(\abs{X}^{2nm})$, а с некоторыми искусственно полученными распределениями --- <<комбинациями>> маргинальных распределений, что позволяет снизить сложность алгоритма до $O\big((nm)^2\big)$.   
\end{enumerate}
