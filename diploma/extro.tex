
В настоящей работе рассмотрены задачи, связанные с  использованием экспертных оценок в форме распределений возможности в рамках теории возможностей Ю.~П.~Пытьева (такие оценки были названы нечёткими). В рамках упомянутой теории были впервые поставлены и решены следующие задачи:
задача принятия решения о выборе наиболее качественных объектов на основе нечётких экспертных данных об их характеристиках, сформулированная как задача на минимум возможности принять неверное решение;
задача нахождения коллективного мнения экспертов, не противоречащего мнениям отдельных экспертов, сформулированная как задача нахождения верхней грани на множестве возможностных распределений (экспертных оценок), заданных экспертами, используя введённое на множестве распределений отношение квазипорядка.ф 
 Приведены алгоритмы решения этих задач, которые благодаря эффективности операций <<$\sup$>> и <<$\inf$>> в качестве сложения и умножения значений возможности соответственно оказываются вычислительно эффективными.