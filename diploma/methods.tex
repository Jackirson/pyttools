% Математические методы: операции с экспертным оценками (сложение, усреднение, сравнение)

\label{sec:math_methods_global}

%Для математического моделирования субъективных суждений и связанной с ними нечёткости разработано множество математических конструкций: субъективная вероятность Сэведжа [Сэведж, 1972]  как мера неуверенности субъекта, суждения которого удовлетворяют определенным условиям «рациональности»; верхние и нижние вероятности Демпстера [Демпстер, 1967] , характеризующие неполноту наблюдений и отражающие неопределенность в теории вероятностей, моделируемую многозначными отображениями; правдоподобие и доверие Шеффера [Шеффер, 1976], обобщающие конструкции Демпстера в теории принятия решений; возможность Заде [Заде, 1978], основанная на его теории нечетких множеств [Заде, 1965],~--- далеко не полный перечень таких работ. Следует отметить также вобравшую в себя своих предшественников работу по теории возможностей [Дюбуа, Прад,1988], и, наконец, теорию возможностей [Пытьев, 2000]. На последней работе остановимся подробнее, посколькупредставленная в ней теория возможностей используется в настоящей работе.

Для анализа экспертных оценок надо определить математические объекты, как выступающие непосредственно в роли моделей экспертных оценок, так и вспомогальные, и определить допустимые математические операции над ними. Как уже говорилось, в качестве модели экспертной оценки числа $x \in X \subset \R$ можно взять разные математические объекты: % (см.  \ref{sec:intro_subjective})
\begin{enumerate}
  \item Одно число из того же множества $X$: $\hat{x} \in X$. Такую модель назовём точечной экспертной оценкой;
  \item Пара действительных чисел $x^{(1)},\, x^{(2)} \in X$, задающая интервал $[x^{(1)},\, x^{(2)}]$. Такую модель назовём интервальной экспертной оценкой;
  \item Заданная таблицей, графиком или иным образом функция $p(\cdot)$, определённая на заданном интервале $[x^{(1)},\, x^{(2)}]$, принимающая значения на отрезке $\zo$, см. рисунок \ref{ris:fuzzy_number_intro}.  Подобный объект, благодаря работам [Заде, Дюбуа], часто называют {\sl нечётким числом}, поэтому назовём такую модель нечёткой экспертной оценкой. В вырожденном случае, если только для одной точки $x_0 \in [x^{(1)},\, x^{(2)}]$ упомянутая функция отлична от нуля, монжно говорить о <<чётком>> или <<дельтообразном>> виде нечёткой оценки. 
\end{enumerate}

Операции над точечными экспертными оценками идентичны операциям над действительными числами. Такие оценки можно обычным образом складывать, вычислять среднюю оценку и сравнивать, пользуясь естественной упорядоченностью числовой прямой. Но в рамках такой модели эксперт не может выразить неуверенность, частичное или полное незнание, возможность нескольких сценариев развития исследуемого объекта. Со вторым и третьим типом экспертных оценок, устраняющими эти недостатки, операции уже сложнее.

\subsection{Интервальные экспертные оценки}

Интервальная математика упоминается c 1966 году в работах Мура~\cite{moore1966interval, moore2009introduction}. Она была разработана, в первую очередь, как база для моделирования неопределённости данных физических измерений при обработке этих данных на ЭВМ. Данным физических измерений присущи погрешности как случайного, так и неслучайного (систематического) характера, и интервальные числа не привязаны к конкретной модели их формирования. Поэтому они подходят и для моделирования неопределённости, присущей экспертным оценкам, если эксперт испытывает и высказывает неуверенность. 

Интервальные экспертные оценки следует рассматривать как заданную экспертом пару не случайных чисел, между которыми, по его мнению, заключён оцениваемый параметр $x \in X$, не известный эксперту на момент выставления оценки.  Интервалы со случайными границами, в т.\,ч. доверительные интервалы, в настоящей работе не рассматриваются.

Сложение двух интервалов $A = [a_1,\,a_2]$ и $B = [b_1,\,b_2]$, а также умножение интервала на число $\lambda \in \R$ определяются у Мура следующим образом:
\begin{equation*}
\begin{split}
 A + B = & [a_1+b_1,\,a_2+b_2];\\
% A \mult B &= [\min\{a_1b_1,a_1b_2,a_2b_1,a_2b_2\},\,\max\{a_1b_1,a_1b_2,a_2b_1,a_2b_2\}]
 \lambda A = & [\lambda a_1, \, \lambda a_2];
\end{split} 
\end{equation*}
С помощью этих операций можно вычилять, например, среднее арифметическое интервальных оценок. 

С другой стороны, Феллер~\cite{cit:feller} предлагает рассматривать интервальные оценки как подмножества числовой приямой, на множестве которых можно ввести отношение частичного порядка <<$\prec$>>, заключающееся, по определению, в следующем:
\begin{equation}
\label{interv_order}
\begin{split}
\forall A: & A \prec A; \\
\forall A, B: & A \prec B \text{ и } B \prec A \Rightarrow A = B; \\
\forall A, B, C: & A \prec B \text{ и } B \prec C \Rightarrow A \prec C.
\end{split}
\end{equation}

Конструктивно, для интервалов $A, B$:  $A \prec B$ тогда и только тогда, когда $A \subset B$. Порядок частичный, поскольку существуют не вложенные друг в друга интервалы. Однако, для любых двух интервалов $A, B$ можно ввести бинарную операцию, сопоставляющую им такое множество $C$, что $C \succ A$ и $C \succ B$, причём всякое другое $C \neq C': C' \succ A, C' \succ B$ будет содержать в себе $C$. Это есть определение точной верхней грани множества всех интервалов, которые включают в себя $A$ и $B$:
\begin{equation}
\label{interv_sup}
 C = A \vee B.
\end{equation}
Часто, для краткости, выражение \eqref{interv_sup} произносят как <<супремум  $A$ и $B$>>. Констуктивно, супремум двух интервалов --- это их объединение $A \cup B$. Аналогично вводится инфинум $A$ и $B$, обозначаемый $A \wedge B$ и констуктивно эквивалентый их пересечению $A \cap B$.

Операции <<$\vee$>> и <<$\wedge$>> содержат глубокий физический смысл: они позволяют найти коллективное экспертное мнение, в модели интевальных оценок, не противоречащее мнению отдельно взятых экспертов. Так, если $A, B$ --- мнения двух экспертов, то $A \vee B$ отражает доверие со сторноы лица, принимающего решение и к тому, и к другому эксперту, т.е. это лицо допускает и варианты развития событий, предсказанные как первым, так и вторым экспертом. Инфинум  $A \wedge B$ отражает, напротив, недоверие к экспертам, оставляя лишь общую часть заданных ими интервалов в качестве коллективного мнения.

\subsection{Нечёткие экспертные оценки и теории возможностей}

\begin{comment}
В случае использования <<интервальных>> оценок эксперт не указывает предпочтительность тех или иных значений $x \in [x_{min},\,x_{max}]$. Если же она указана, получается оценка нового типа --- {\sl нечёткая} экспертная оценка. Эти более сложные оценки могут моделироваться разными способами, из которых мы рассмотрим два --- теорию возможностей (т.\,в.) Л.~А.~Заде \cite{3} и т.\,в. Ю.~П.~Пытьева \cite{4}. Но сначала введём общие понятия.

Будем говорить, что задана нечёткая экспертная оценка, если для всех $x \in X$ экспертом задана величина $\p_{\tilde x}(x) \in \zo$, выражающая относительную возможность\footnote{Хотя в быту в таком контексте часто говорят <<вероятность>>, во всех математических теориях и в настоящей работе важно подчеркнуть нестохастический характер субъективной информации, см. Введение. В некоторых работах, например, \cite{4_1}, используется термин <<субъективная вероятность>>.} того, что оцениваемый параметр $\tilde x$ примет значение $x$.
Иными словами, задана функция $\p_{\tilde x}(\cdot): X \rightarrow \zo$, называемая функцией распределения возможностей значений $x$ параметра $\tilde x$. При этом $\p_{\tilde x}(x) = 1$ соотвествует наиболее возможным $x$, а $\p_{\tilde x}(x) = 0$ --- принципиально невозможным, по мнению эксперта, значениям $\tilde x$. Пример нечёткой оценки приведён на рис. \ref{ris:fuzzy_ass00}\footnote{На рис. \ref{ris:fuzzy_ass00} функция $\p_{\tilde x}(\cdot)$ изображена непрерывной. Вообще говоря, задання экспертом функция $\p'_{\tilde x}(\cdot)$ не обязана быть непрерывной. В т.\,в. Заде $\p_{\tilde x}(\cdot)$ считается непрерывной, подразумевая, что $\p'_{\tilde x}(\cdot)$ дополняется по непрерывности до $\p_{\tilde x}(\cdot)$. В т.\,в. Пытьева $\p_{\tilde x}(\cdot)$ считается измеримой, но не обязана быть непрерывной.}.

\begin{figure}[h]
\center{\includegraphics[width=0.7\linewidth]{fuzzy_ass00}}
\caption{\small Пример нечёткой оценки. По горизонтальной оси отложены значения $x$ оцениваемого параметра $\tilde x$. По вертикальной оси отложена возможность того, что $\tilde x = x$. Наиболее возможным эксперт признал значение $6$, менее возможными ---  $\{4,5,7,8\}$, и совсем невозможными --- $\{1,2,3,9,10\}$.}
\label{ris:fuzzy_ass00}
\end{figure}

Нечёткая оценка позволяет эксперту выразить как точное знание $\tilde x$, положив $\tilde x = x_0, x_0 \in X$ единственно возможным (рис. \ref{ris:fuzzy_ass01}, слева), так и полное незнание $\tilde x$, положив все $x \in X$ равновозможными (рис. \ref{ris:fuzzy_ass01}, справа), а также любой промежуточный случай.

\begin{figure}[h]
\center{\includegraphics[width=0.35\linewidth]{fuzzy_ass01a}
        \includegraphics[width=0.35\linewidth]{fuzzy_ass01b} }
\caption{\small Нечёткие оценки, выражающие точное знание (слева) и полное незнание (справа). }
\label{ris:fuzzy_ass01}
\end{figure}

\subsection{Теория возможностей Л.~А.~Заде как средство моделирования нечётких экспертных оценок}

Т.\,в. Заде основана на теории нечётких множеств \cite{5,6}. Связь между этими теориями следующая. Пусть нечёткое множество $F \subseteq X$ с функцией принадлежности $\mu_{F}(\cdot): X \rightarrow \zo$ является нечётким ограничением \cite{3} оцениваемого параметра $\tilde x$. Например, $\tilde x$ --- возраст человека, $F$ --- утверждение <<этот человек --- молодой>>, а $\mu_{F}(x)$ интерпретируется как <<степень соответствия>> возраста $x$ утверждению $F$. Тогда по определению полагается $\p_{\tilde x}(\cdot) \defeq \mu_{F}(\cdot)$. В нашем примере это означает, что возможность того, что возраст человека окажется равным $x$, если эксперт утверждает, что <<человек молодой>>, численно равна <<степени соответствия>> этого возраста нечёткому множеству <<молодость>>. 

В теории нечётких множеств вводятся операции над нечёткими множествами $A, B$ в терминах их функций принадлежности: 
\begin{equation*}
  \mu_{A \cup B}(\cdot) \defeq \sup\{\mu_{A}(\cdot),\mu_{B}(\cdot)\}; \ \ \mu_{A \cap B}(\cdot) \defeq \inf\{\mu_{A}(\cdot),\mu_{B}(\cdot)\}; 
\end{equation*}
Аналогично арифметическим действиям, которые можно выполнять с одиночнымии числами, вводятся бинарные операции $f(A, B)$, где $f(\cdot, \cdot): \R \times \R \rightarrow \R$ --- любая числовая функция, например, $\plus$ или $\mult$ \cite{7}: 
\begin{equation*}
  \mu_{f(A, B)}(z) = \underset{(x, y): f(x, y) = z}\sup \inf\{\mu_{A}(x),\mu_{B}(y)\}, \ x,y,z \in X; 
\end{equation*}
Поскольку $\p_{\tilde x}(\cdot) \defeq \mu_{F}(x)$, то и для функций $\p_{\tilde x}(\cdot)$, т.\,е. для нечётких экспертных оценок, в т.\,в. Заде используются арифметические действия. Более того, и это принципиально важно: содержательно интерпретируются все конкретные значения $\p_{\tilde x}(\cdot)$, например, $0.5, 0.2$. Такой подход имеет следующий недостаток.

Когда эксперт выставляет оценку в виде балла, например, $X = \{1, 2, ..., 10\}$, существует специальный регламент относительно того, какой смысл несёт каждый балл $x \in X$. Каждый случай проведения экспертизы требует разработки регламента и установления соглашения между экспертами. Этот процесс постоянно совершенствуется, но до сих пор в таких регламентах остается немало неопределенности. Представим теперь, что необходимо договориться не только по поводу смысла значений $x \in X$, но и по поводу смысла значений $\p_{\tilde x}(\cdot) \in \zo$. Это создаст неподъемную нагрузку на экспертов. 

Этот недостаток устранён в рамках т.\,в. Пытьева.
\end{comment}
\subsection{Теория возможностей Ю.~П.~Пытьева как средство моделирования нечётких экспертных оценок}
\label{sec:math_methods_ours}
\begin{comment}
Т.\,в. Ю.~П.~Пытьева построена аксиоматически, аналогично построению теории вероятностей \cite{4,8}. Пусть имеется множество элементарных исходов $\Om$ и задана $\sigma$-алгебра событий $\Alg \subseteq 2^{\Om}$. Возможностью называется всякая измеримая функция $\P(\cdot): \Alg \rightarrow \zo$, обладающая определенными свойствами, которые будут определены ниже. Тройка $\OAP$ называется пространством с возможностью. 

Определим шкалу $\scL$ значений возможности $\P(\cdot)$ как интервал $\zo$ с естественной упорядоченностью $\leqslant$ и двумя бинарными операциями, $\plus: \zo \times \zo \rightarrow \zo$ и $\mult: \zo \times \zo \rightarrow \zo$. С точки зрения эксперта, $\scL$ --- это правила, по которым он размещает возможности событий из $\Alg$ на отрезке $\zo$. 

Пусть $\Gamma$ --- группа строго монотонно возрастающих непрерывных $\gamma(\cdot): \zo \rightarrow \zo,\ \gamma(0)=0,\ \gamma(1)=1$ с групповой операцией $\circ: f \circ g = f(g(\cdot))$. Возможности $\P$ и $\P'$ назовем {\sl эквивалентными}, если $\P' = \gamma(\P)$. Эквивалентные возможности образуют класс возможностей $\PP$ и соотвествующий класс пространств с возможностью \{$\OAP\ |\ \P \in \PP $\}. Принцип относительности (см. \cite{8}) гласит, что утверждения и модели, полученные с использованием возможности, содержательны в том и только в том случае, если они остаются верными при замене данной возможности $\P$ на любую эквивалентную ей $\P'$. 
%Иными словами, не важен выбор конкретной шкалы $\scL$, потому что группа $\Gamma$ порождает группу $\tilde \Gamma$ автоморфизмов $\scL$.

Из принципа относительности следует, что операции $\plus$ и $\mult$ над значениями возможностей двух произвольных множеств $A, B \in \Alg$, т.е. числами $a = \P(A) \in \zo, b = \P(B) \in \zo$, должны удовлетворять условиям:
\begin{center}
  $\gamma(a \plus b) = \gamma(a) \plus \gamma(b),\ $
  $\gamma(a \mult b) = \gamma(a) \mult \gamma(b) $
\end{center}
Также эти операции должны быть коммутативны и не выводить за пределы $\zo$:
\begin{center}
  $a \plus b = b \plus a,\ a \plus 1 = 1,\ a \plus 0 = a, $\\
  $a \mult b = b \mult a,\ a \mult 1 = a,\ a \mult 0 = 0, $
\end{center}
В \cite{8} показано, что единственный способ получить все эти свойства в классе непрерывных операций --- определить
\begin{equation*}
 a \plus b \defeq \sup\{a,b\}; \ \  a \mult b \defeq \inf\{a,b\} 
\end{equation*}
Эти операции естественным образом распространяются на конечное или счётное число операндов. 

Таким образом, возможность --- мера, принимающая значения на шкале $\scL = \scale$, где сложение понимается как точная верхняя грань, а умножение --- как точная нижная грань. По аналогии с вероятностной мерой постулируются два свойства возможности:
\begin{enumerate}
 \item аддитивность: $\P(A \cup B) = \P(A) \plus \P(B)\ \forall A,B \in \Alg$, где $a \plus: b \defeq \sup\{a,b\}, a, b \in \zo$;\footnote{Из-за переопределенной операции сложения в свойстве (1), в отличие от аналогичного свойства вероятностей, не требуется $A \cup B = \o{}$.}
 \item нормировка на единицу: $\P(\Om) = \underset{\omega \in \Om}\sup\ \P({\omega}) = 1$;
\end{enumerate}

Моделью оцениваемого параметра в т.\,в. Пытьева выступает математический объект, аналогичный понятию случайной величины в теории вероятностей --- нечёткий элемент $\tilde x(\cdot): \Om \rightarrow X$. Моделью нечёткой экспертной оценки служит функция распределения возможностей нечёткого элемента $\p_{\tilde x}(\cdot) \defeq \P(\{\tilde x(\omega) = x\})$. 

Итак, принципиальное отличие т.\,в. Пытьева от т.\,в. Заде в том, что в ней содержательно интерпретиуются не все конкретные значения $\p_{\tilde x}(\cdot)$, а только {\sl упорядоченность} этих значений, и отдельно --- неподвижная относительно группы $\Gamma$ точка <<0>>, так что $\p_{\tilde x}(x_0) = 0$ означает принципиальную невозможность значения $x_0 \in X$. Используя такую модель экспертной оценки, можно предоставить эксперту свободу выбора своей субъективной шкалы $\scL$, и экспертные оценки на рис. \ref{ris:fuzzy_ass03} будут выражать одно и то же экспертное мнение.

\begin{figure}[h]
\center{\includegraphics[width=0.7\linewidth]{fuzzy_ass03}}
\caption{\small Пример двух эквивалентных экспертных оценок.}
\label{ris:fuzzy_ass03}
\end{figure}

\end{comment}
