% Математические методы: операции с экспертным оценками (сложение, усреднение, сравнение)

\label{sec:math_methods_global}

%Для математического моделирования субъективных суждений и связанной с ними нечёткости разработано множество математических конструкций: субъективная вероятность Сэведжа [Сэведж, 1972]  как мера неуверенности субъекта, суждения которого удовлетворяют определенным условиям «рациональности»; верхние и нижние вероятности Демпстера [Демпстер, 1967] , характеризующие неполноту наблюдений и отражающие неопределенность в теории вероятностей, моделируемую многозначными отображениями; правдоподобие и доверие Шеффера [Шеффер, 1976], обобщающие конструкции Демпстера в теории принятия решений; возможность Заде [Заде, 1978], основанная на его теории нечетких множеств [Заде, 1965],~--- далеко не полный перечень таких работ. Следует отметить также вобравшую в себя своих предшественников работу по теории возможностей [Дюбуа, Прад,1988], и, наконец, теорию возможностей [Пытьев, 2000]. На последней работе остановимся подробнее, посколькупредставленная в ней теория возможностей используется в настоящей работе.

Для анализа экспертных оценок надо определить математические объекты, как выступающие непосредственно в роли моделей экспертных оценок, так и вспомогальные, и определить допустимые математические операции над ними. Как уже говорилось, в качестве модели экспертной оценки числа $x \in X \subset \R$ можно взять разные математические объекты: % (см.  \ref{sec:intro_subjective})
\begin{enumerate}
  \item Одно число из того же множества $X$: $\hat{x} \in X$. Такую модель назовём точечной экспертной оценкой;
  \item Пара действительных чисел $x^{(1)},\, x^{(2)} \in X$, задающая интервал $[x^{(1)},\, x^{(2)}]$. Такую модель назовём интервальной экспертной оценкой;
  \item Заданная таблицей, графиком или иным образом функция $p(\cdot)$, определённая на заданном интервале $[x^{(1)},\, x^{(2)}]$, принимающая значения на отрезке $\zo$, см. рисунок \ref{ris:fuzzy_number_intro}.  Подобный объект, благодаря работам \cite{citeZadeh, dubois_prade-1990}, часто называют {\sl нечётким числом}, поэтому назовём такую модель нечёткой экспертной оценкой. В вырожденном случае, если только для одной точки $x_0 \in [x^{(1)},\, x^{(2)}]$ упомянутая функция отлична от нуля, монжно говорить о <<чётком>> или <<дельтообразном>> виде нечёткой оценки. 
\end{enumerate}

Операции над точечными экспертными оценками идентичны операциям над действительными числами. Такие оценки можно обычным образом складывать, вычислять среднюю оценку и сравнивать, пользуясь естественной упорядоченностью числовой прямой. Но в рамках такой модели эксперт не может выразить неуверенность, частичное или полное незнание, возможность нескольких сценариев развития исследуемого объекта. Со вторым и третьим типом экспертных оценок, устраняющими эти недостатки, операции уже сложнее.

\subsection{Интервальные экспертные оценки}

Интервальная математика упоминается c 1966 года в работах Мура~\cite{moore1966interval, moore2009introduction}. Она была разработана, в первую очередь, как база для моделирования неопределённости данных физических измерений при обработке этих данных на ЭВМ. Данным физических измерений присущи погрешности как случайного, так и неслучайного (систематического) характера, и интервальные числа не привязаны к конкретной модели их формирования. Поэтому они подходят и для моделирования неопределённости, присущей экспертным оценкам, если эксперт испытывает и высказывает неуверенность. 

Интервальные экспертные оценки следует рассматривать как заданную экспертом пару не случайных чисел $a_2 \leq a_1$, между которыми, по его мнению, заключён оцениваемый параметр $x \in X$, не известный эксперту на момент выставления оценки.  Интервалы со случайными границами, в т.\,ч. доверительные интервалы, в настоящей работе не рассматриваются.

Сложение двух интервалов $A = [a_1,\,a_2]$ и $B = [b_1,\,b_2]$, а также умножение интервала на число $\lambda \in \R$ определяются у Мура следующим образом:
\begin{equation*}
\begin{split}
 A + B = & [a_1+b_1,\,a_2+b_2];\\
% A \mult B &= [\min\{a_1b_1,a_1b_2,a_2b_1,a_2b_2\},\,\max\{a_1b_1,a_1b_2,a_2b_1,a_2b_2\}]
 \lambda A = & [\lambda a_1, \, \lambda a_2];
\end{split} 
\end{equation*}
С помощью этих операций можно вычислять, например, среднее арифметическое интервальных оценок. 

С другой стороны, Феллер~\cite{cit:feller} предлагает рассматривать интервальные оценки как подмножества числовой приямой, на множестве которых можно ввести отношение частичного порядка <<$\prec$>>, заключающееся, по определению, в следующем:
\begin{equation}
\label{interv_order}
\begin{split}
\forall A: & A \prec A; \\
\forall A, B: & A \prec B \text{ и } B \prec A \Rightarrow A = B; \\
\forall A, B, C: & A \prec B \text{ и } B \prec C \Rightarrow A \prec C.
\end{split}
\end{equation}

Конструктивно, для интервалов $A, B$:  $A \prec B$ тогда и только тогда, когда $A \subset B$. Порядок частичный, поскольку существуют не вложенные друг в друга интервалы. Однако, для любых двух интервалов $A, B$ можно ввести бинарную операцию, сопоставляющую им такое множество $C$, что $C \succ A$ и $C \succ B$, причём всякое другое $C' \neq C: C' \succ A, C' \succ B$ будет содержать в себе $C$. Это есть определение точной верхней грани множества всех интервалов, которые включают в себя $A$ и $B$:
\begin{equation}
\label{interv_sup}
 C = A \vee B.
\end{equation}
Часто, для краткости, выражение \eqref{interv_sup} произносят как <<супремум  $A$ и $B$>>. Конструктивно, супремум двух интервалов --- это их объединение $A \cup B$. Аналогично вводится инфинум $A$ и $B$, обозначаемый $A \wedge B$ и конструктивно эквивалентный их пересечению $A \cap B$.

Операции <<$\vee$>> и <<$\wedge$>> содержат глубокий физический смысл: они позволяют найти коллективное экспертное мнение, в модели интервальных оценок, не противоречащее мнению отдельно взятых экспертов. Так, если $A, B$ --- мнения двух экспертов, то $A \vee B$ отражает доверие со стороны лица, принимающего решение и к тому, и к другому эксперту, т.е. это лицо допускает и варианты развития событий, предсказанные как первым, так и вторым экспертом. Инфинум  $A \wedge B$ отражает, напротив, недоверие к экспертам, оставляя лишь общую часть заданных ими интервалов в качестве коллективного мнения.

\subsection{Нечёткие экспертные оценки и теории возможностей}

В случае использования <<интервальных>> оценок эксперт не указывает {\sl предпочтительность} тех или иных значений $x \in [x^{(1)},\, x^{(2)}]$. Если же она указана, получается оценка третьего типа --- нечёткая экспертная оценка. Эти более сложные оценки могут моделироваться с помощью упомянутых в разделе~\ref{sec:intro_decision} теорий нечётких множеств и теории возможностей.
%\cite{citeZadeh, dempster, tahani, dubois_prade-1990}

С момента своего появления в тех же 70-х годах прошлого века, теории возможности изначально предназначены для моделирования неопределённости субъективного характера, связанной с принципиальной невозможностью точного определения множества исследуемых объектов (например, множества высоких людей) или с неполнотой знаний об исследуемом объекте, выраженных экспертными оценками. Это принципиально отличает данные теории от теории вероятностей, несмотря на внешнюю схожесть математических аппаратов этих теорий. 

Возможность (или возможностная мера) определяется в теории возможностей Заде \cite{citeZadeh, dubois_prade-1990} аналогично тому, как в теории вероятностей определяется вероятность. Пусть $\Omega$~--- множество элементарных событий, и $\Po(\Omega)$~--- $\sigma$-алгебра всех подмножеств множества $\Omega$.
\begin{definition}
\label{def_possibility}
\emph{Возможностью} называется всякая функция $\P:\ \Po(\Omega)\to[0,1]$, удовлетворяющая следующим аксиомам:
\begin{compactenum}
\item $\P(\varnothing) = 0$,\label{axiom_P1}
\item $\P(\Omega) = 1$,\label{axiom_P2}
\item $\P\Big(\bigcup\limits_{A\in\Alg} A\Big) = \sup\limits_{A\in\Alg}\P(A)$, для любого множества событий $\Alg\subset\Po(\Omega)$.\label{axiom_P3}
\end{compactenum}
\end{definition}

Аксиомы возможности аналогичны аксиомам вероятности $\Pr$~\cite{kolmogorov}, но математические операции~--- другие. Вместо обычного сложения в теории возможностей используется операция <<$\max$>> (в случае бесконечного числа слагаемых~--- <<$\sup$>>), а вместо умножения --- операция <<$\min$>> (в бесконечном случае~--- <<$\inf$>>). Эти операции не выводятся, а постулируются в рамках теории возможностей Заде, однако в них есть определённый математический смысл, который можно пояснить на примере.
\begin{example}
Пусть множество элементарных событий есть множество возрастов людей в годах, $\Om = \{0, 1, \ldots, 110\}$, а возможность каждого элементарного события есть возможность того, что наперёд фиксированному человеку --- именно столько лет, с точки зрения эксперта, который оценивает возраст этого человека.  Тогда логично, что возможность того, что возраст человека лежит в интервале от $20$ до $40$ лет, определяется тем возрастом из этого промежутка, который наиболее правдоподобен. 

С другой стороны, если есть ещё один человек, возраст которого оценивается, то логично, что возможность утверждения <<возраст {\sl обоих} этих людей лежит в указанном интервале>> определяется (вообще говоря, {\sl в основном}, но в рассматриваемой модели --- {\sl только}) тем, насколько возможно это для того человека, для которого это утверждение менее правдоподобно. 
\end{example}

\begin{notice}
Как уже говорилось, теория возможностей Заде основана на его теории нечётких множеств \cite{ZadehPrime}. Связь между этими теориями следующая. Пусть нечёткое множество $F \subseteq X$ с функцией принадлежности $\mu_{F}(\cdot): X \rightarrow \zo$ является нечётким ограничением \cite{citeZadeh} оцениваемого параметра, моделируемого числом $x$. Например, $x$ --- возраст человека, $F$ --- утверждение <<этот человек --- молодой>>, а $\mu_{F}(x)$ интерпретируется как <<степень соответствия>> возраста $x \in X$ утверждению $F$. Тогда по определению полагается $\P(\cdot) \defeq \mu_{F}(\cdot)$. В нашем примере это означает, что возможность того, что возраст человека окажется равным $x$, если эксперт утверждает, что <<человек молодой>>, численно равна <<степени соответствия>> этого возраста нечёткому множеству <<молодость>>. 
\end{notice}
\begin{notice}
Вместо равенства $\Pr(A) + \Pr(\Omega\setminus A) = 1$, устанавливающего однозначную связь между вероятностью всякого события $A\in\Po(\Omega)$ и его отрицанием, в теории возможностей получаем
\begin{equation}
\label{max_PA_PnotA}
    \max\{\P(A), \P(\Omega\setminus A)\} = 1.
\end{equation}
Равенство~\eqref{max_PA_PnotA} интерпретируется как факт, состоящий в том, что из двух противоположных событий по крайней мере одно безусловно возможно. В то же время оно не определяет однозначной связи между $\P(A)$ и $\P(\Omega\setminus A)$, что согласуется со стремлением при моделировании субъективных суждений не устанавливать жёсткой связи между показателями, свидетельствующими в пользу некоторого события, и показателями, свидетельствующими против него~\cite{dubois_prade-1990}.
\end{notice}

В качестве модели нечёткой оценки возможность Заде можно использовать следующим образом. Возьмём некий интервал $A \subset X$  (интервальную оценку) в качестве множества элементарных событий (точек $x \in А$), на которых эксперт задаёт функцию $p(\cdot): A \rightarrow [0,1]$, причём он обязательно выбирает точку $x_0$, для которой $p(x_0) = 1$. Тогда этой функции можно однозначно сопоставить возможность $\P(\{x\}) \define= p(x) \in \zo,\; x\in A$ с введёнными для возможности математическими операциями.    

Если эксперт (или эксперты) задаёт $n$ оценок на, возможно, различных множествах $A_1, \ldots, A_n$, следует выбрать в качестве множества элементарных событий всё множество $X$, а функцию $p_i$, указывающую предпочтительность значений на множестве $A_i,\; i \in \setN$ продолжить вне этого множества на $X$ значениями $p_i(x) = 0, x \notin A_i$. Таким образом, $A_i$ становится носителем $p_i$, а также носителем порождаемой последней функцией возможностной меры $\P_i$. Носитель $p_i$ обозначается символом $\supp\; p_i$.

Существуют, однако, фундаментальные проблемы в рамках теории возможностей Заде и её аналогов:
\begin{enumerate}
\item
Операции сложения (<<$\max$>>) и умножения (<<$\min$>>) значений возможности звучат логично, но ниоткуда не выводятся.
\item
Не прослеживается связь теории возможностей, призванной моделировать субъективные суждения, и математической статистики, призванной моделировать случившиеся факты, хотя эта связь должна быть установлена, когда факты имеются в наличии.
\item
Когда эксперт выставляет оценку $x \in X$ (первого типа) в виде балла, например, $X = \{1, 2, ..., 10\}$, существует специальный регламент относительно того, какой смысл несёт каждый балл $x \in X$. Каждый случай проведения экспертизы требует разработки регламента и установления соглашения между экспертами. Этот процесс постоянно совершенствуется, но до сих пор в таких регламентах остается немало неопределенности. Представим теперь, что необходимо договориться не только по поводу смысла значений $x \in X$, но и по поводу смысла значений $\p(\cdot) \in \zo$ при выставлении оценки третьего типа, что требуется для сохранения содержательной интерпретации каждого конкретного числа отрезка $\zo$ всеми экспертами одновременно. Это создаст неподъемную нагрузку на экспертов и разработчиков экспертизы и является главным недостатком теории возможностей Заде. 
\end{enumerate}
Эти недостатки устранены в рамках теории возможностей Ю.~П.~Пытьева.

\subsection{Теория возможностей Ю.~П.~Пытьева как средство моделирования нечётких экспертных оценок}
\label{sec:math_methods_ours}

В конце 1990-ых~--- начале 2000-ых профессором Московского университета Ю.\,П.\;Пытьевым предложен новый вариант теории возможностей~\cite{possbook, cit:smf, possbook2, probbook, pytyev_experts}. Этот вариант принципиально отличается от теории возможностей Заде и её аналогов наличием \emph{принципа относительности}, который при сохранении сформулированных выше аксиом возможности требует инвариантности любых выводов, полученных в рамках теории, относительно группы монотонно возрастающих непрерывных преобразований $\vtheta(\cdot):\ [0,1]\to[0,1]$, значений возможности: $\vtheta(0) = 0,\ \vtheta(1) = 1$. При моделировании экспертных оценок данный принцип позволяет каждому эксперту задавать значения возможностей в своей собственной субъективной шкале, заботясь лишь о правильной (с точки зрения эксперта) упорядоченности возможностей событий.

\begin{comment}
При этом в отличие от теории возможностей Заде и её аналогов теория возможностей Пытьева изначально разработана как альтернатива теории вероятностей при моделировании стохастической случайности.
Ю.\,П.\;Пытьевым показано, что одна и та же возможностная модель позволяет моделировать стохастический эксперимент, вероятностная модель которого может изменяться от испытания к испытанию.  Им дана частотная интерпретация возможности, состоящая в том, что в достаточно длинной серии независимых испытаний частота более возможного события почти наверное больше частоты менее возможного. Данный факт послужил причиной того, что в настоящей диссертации за основу при построении возможностной модели случайной формы изображения взята теория возможностей в варианте Ю.\,П.\;Пытьева.

\subsubsection{Возможность как мера относительной предопредлённости исходов стохастического эксперимента}
\label{sec:sec_20151029_01}

Рассмотрим стохастический эксперимент, моделью которого является вероятностное пространство ${(\Omega,\, \Alg,\, \Pr)}$. В~\cite{possbook, possbook2} возможностная модель этого эксперимента определяется исходя из того, что возможность всякого события ${A\in\Alg}$ должна служить мерой его относительной (относительно других событий) предопределённости. В связи с этим возможность $\P$ задаётся как функция, определённая на $\Alg$, принимающая значения на отрезке $[0,1]$ и удовлетворяющая следующим условиям:
\begin{enumerate}
\item\label{item_poss_rel_1}
    $\P(\varnothing) = 0$.
\item\label{item_poss_rel_2}
    $\P(\Omega) = 1$.
\item\label{item_poss_rel_3}
    Для любых $A,\, B\in\Alg$ если $\Pr(A)\leqs\Pr(B)$, то $\P(A)\leqs\P(B)$.
\end{enumerate}

Условия \ref{item_poss_rel_1}) и \ref{item_poss_rel_2}), приведённые выше, носят формальный характер и  интерпретируются следующим образом:
\begin{enumerate}
\item
    $\Omega$~--- множество всевозможных элементарных исходов рассматриваемого стохастического эксперимента. Событие, состоящее в том, что в результате эксперимента не реализуется ни один из элементарных исходов $\omega\in\Omega$, невозможно (его возможность равна нулю).
\item
    Событие, состоящее в том, что в результате рассматриваемого стохастического эксперимента реализуется один из элементарных исходов $\omega\in\Omega$, достоверно (его возможность равна единице).
\end{enumerate}

В свою очередь условие \ref{item_poss_rel_3}) выражает естественное стремление определить возможность $\P$ так, чтобы более вероятные события были одновременно и более возможными. При этом на возможность $\P$ не налагается никаких других требований, что приводит нас к следующему принципу, занимающему центральное место в теории возможностей Пытьева.

\end{comment}

\subsubsection{Принцип относительности}
\label{sec:sec_20151029_02}

Обозначим $\bTheta$ множество всех монотонно возрастающих непрерывных функций $\vtheta(\cdot):\ [0,1]\to[0,1]$, таких что $\vtheta(0) = 0$, $\vtheta(1) = 1$.
Согласно \emph{принципу относительности} возможности $\P$ и $\P'$ считаются эквивалентными, если найдётся такая функция $\vtheta(\cdot)\in\bTheta$, что $\P(A) = \vtheta(\P'(A)),\ A\in\Alg$, а утверждения и выводы, сформулированные в рамках возможностной модели, определяемой возможностью $\P$, считаются содержательными, если и только если они остаются верными при замене $\P(\cdot)$ на $\P'(\cdot) = \vtheta(\P(\cdot))$ для любой функции $\vtheta(\cdot)\in\bTheta$.

Поясним сформулированный принцип на примере двух утверждений:
\begin{itemize}
\item
    \emph{Возможность $\P(A)$ события $A$ равна $0.9$.}\\
    Данное утверждение в теории возможностей не является содержательным, так как найдётся такая функция $\vtheta(\cdot)\in\bTheta$, что величина $\vtheta(\P(A))=\vtheta(0.9)$ не будет равна $0.9$.

\item
    \emph{Возможность события $A$ больше, чем возможность события $B$ (событие $A$ более возможно, чем $B$).}\\
    Такое утверждение, в отличие от предыдущего, может быть содержательно истолковано, так как для любой функции $\vtheta(\cdot)\in\bTheta$ из соотношения $\P(A)>\P(B)$ следует $\vtheta(\P(A))>\vtheta(\P(B))$.
\end{itemize}

Эти примеры демонстрируют характерную особенность теории возможностей, являющуюся прямым следствием принципа относительности: никакие значения возможностей кроме $0$ и $1$ не являются информативными, содержательно может быть истолкована лишь упорядоченность возможностей событий. Этот факт позволяет задать возможность $\P$ в возможностной модели стохастического эксперимента исходя лишь из представления о том, какие события являются более возможными, а какие~--- менее возможными.

\subsubsection{Шкала значений возможности}
\label{sec:sec_20151029_03}

В~\cite{possbook, possbook2} доказана следующая теорема.
\begin{theorem}
\label{th:plus_mult_operations}
Если бинарные операции $\plus$ и $\mult$ являются ассоциативными и взаимно дистрибутивными непрерывными отображениями $[0,1]\times[0,1]\to[0,1]$, и для любых $a,\, b\in[0,1]$, $\vtheta(\cdot)\in\bTheta$
\begin{gather*}
    a\plus b = b\plus a, \quad a\mult b = b\mult a,\\
    0\plus a = a, \quad 0\mult a = 0, \quad 1\plus a = 1, \quad 1\mult a = a,\\
    \vtheta(a\plus b) = \vtheta(a) \plus \vtheta(b); \quad \vtheta(a\mult b) = \vtheta(a) \mult \vtheta(b),
\end{gather*}
то операции $\plus$ и $\mult$ суть
\begin{equation}
\label{20151029_01}
    a \plus b = \max\{a,\, b\}, \quad a \mult b = \min\{a,\, b\}, \quad a,\,b \in [0,1].
\end{equation}
\end{theorem}

Из сформулированного выше принципа относительности возможности и теоремы~\ref{th:plus_mult_operations} следует, что какова бы ни была возможность ${\P:}\ {\Alg\to[0,1]}$, для того, чтобы сделанные на её основе выводы были содержательны, при построении этих выводов значения $\P(A)$ возможностей событий $A\in\Alg$ могут обрабатываться лишь с использованием следующих операций:
\begin{itemize}
\item
    отношения линейного порядка <<$\leqs$>>,
\item
    арифметических операций $\plus$ и $\mult$, определённых согласно~\eqref{20151029_01}.
\end{itemize}

В связи с этим шкалой $\scL = \scale$ значений возможности называется отрезок $[0,1]$ с определёнными на нём упорядоченностью <<$\leqs$>>, понимаемой как обычная упорядоченность действительных чисел, операцией сложения $\plus$, понимаемой как <<$\max$>>, и операцией умножения $\mult$, понимаемой как <<$\min$>>.
\begin{comment}
Пусть ${\{a_1,\, a_2,\, \ldots\}\subset\scL}$~--- последовательность элементов $\scL$. Нижний $\liminf\limits_{i\to\infty} a_i$ и верхний $\limsup\limits_{i\to\infty} a_i$ пределы этой последовательности определяют следующим образом:
\begin{gather*}
    \dst\liminf_{i\to\infty} a_i = \sup_{n=1,\, 2,\, \ldots}\inf_{i\geqs n}a_i,\\
    \dst\limsup_{i\to\infty} a_i = \inf_{n=1,\, 2,\, \ldots}\sup_{i\geqs n}a_i.
\end{gather*}
Если $\liminf\limits_{i\to\infty} a_i = \limsup\limits_{i\to\infty} a_i$, последовательность $\{a_1,\, a_2,\, \ldots\}$ называется сходящейся, а её пределом называется элемент
$$\lim_{i\to\infty} a_i \defeq \liminf_{i\to\infty} a_i = \limsup_{i\to\infty} a_i.$$
\end{comment}

\subsubsection{Возможностная мера}

Пусть $\Omega$~--- множество элементарных событий, и $\Po(\Omega)$~--- $\sigma$-алгебра всех подмножеств множества $\Omega$.
Для формального определения возможности (возможностной меры) $\P:\ \Po(\Omega)\to[0,1]$ в теории возможностей Пытьева введём аксиомы, аналогичный аксиомам теории вероятностей:
\begin{enumerate}
\item\label{item_poss_rel_1}
    $\P(\varnothing) = 0$.
\item\label{item_poss_rel_2}
    $\P(\Omega) = 1$.
\end{enumerate}
и дополнительную аксиому аддитивности, которая, в силу выводов, сформулированных в разделах~\ref{sec:sec_20151029_02}--\ref{sec:sec_20151029_03}, может быть задана единственным образом~--- с использованием в качестве операции суммирования операции $\plus$, определённой как <<$\max$>> (в бесконечно случае~--- <<$\sup$>>), то есть:
\begin{enumerate}
\setcounter{enumi}{2}
\item $\P\Big(\bigcup\limits_{A\in\Alg} A\Big) = \sup\limits_{A\in\Alg}\P(A)$, для любого множества событий $\Alg\subset\Po(\Omega)$.
\end{enumerate}
Отсутствие требования несовместности событий, входящих в $\Alg'$ связано с тем, что из условия $\P(A\cup B) = \max\{\P(A),\P(B)\}$ для любых несовместных $A,\, B\in\Po(\Omega)$ (то есть таких, что $A\cap B=\varnothing$) следует $\P(A\cup B) = \max\{\P(A),\P(B)\}$ для любых $A,\, B\in\Po(\Omega)$, в том числе таких, что $A\cap B\ne\varnothing$.

Таким образом, система аксиом возможности в теории возможностей Пытьева приобретает вид, идентичный аксиомам возможности в теории возможностей Заде~\ref{axiom_P1}--\ref{axiom_P3} на странице \pageref{axiom_P1}. В теории возможностей всякое множество $A\in\Po(\Omega)$ называется событием, % \emph{Возможность} (или возможностная мера) в теории возможностей Пытьева определяется на $\Po(\Omega)$ аксиомами~\ref{axiom_P1_Pyt})--\ref{axiom_P3_Pyt}).
$\P$~--- возможностью, определённой на $\Po(\Omega)$ с помощью аксиом 1--3, а тройка $(\Omega,\, \Alg,\, \P)$ называется пространством с возможностью. %а эксперимент, моделью которого оно является, называется \emph{нечётким}. % При этом как нечёткий может рассматриваться также и стохастический эксперимент, вероятностная модель $(\Omega,\, \Alg,\, \Pr)$ которого связана с возможностной моделью $(\Omega,\, \Alg,\, \P)$ условием~\ref{item_poss_rel_3}) на стр.~\pageref{item_poss_rel_3}.
\begin{comment}
\begin{compactenum}
\item\label{axiom_P1_Pyt} $\P(\varnothing) = 0$,
\item\label{axiom_P2_Pyt} $\P(\Omega) = 1$,
\item\label{axiom_P3_Pyt} $\P\Big(\bigcup\limits_{A\in\Alg} A\Big) = \sup\limits_{A\in\Alg}\P(A)$, для любого множества событий $\Alg\subset\Po(\Omega)$.
\end{compactenum}
\end{comment}



\subsubsection{Нечёткий элемент и нечёткая экспертная оценка}

\begin{comment}

\subsubsection{Частотная интерпретация возможности}

Пространство с возможностью $(\Omega,\, \Po(\Omega),\, \P)$, вообще говоря, служит моделью не одного стохастического эксперимента, а множества экспериментов, вероятностные модели которых принадлежат классу
\begin{equation}
\label{eq:20150821_01}
    \big\{(\Omega,\, \Alg,\, \Pr) \;\big|\ \Pr\in\PPr(\P) \big\}.
\end{equation}
Здесь $\sigma$-алгебра $\Alg$ подмножеств $\Omega$ может совпадать с $\Po(\Omega)$, а может быть её $\sigma$-подалгеброй\footnote{Как правило, чем шире $\Alg$, тем уже класс $\PPr(\P)$. В частности, для произвольной возможности $\P$ при $\Alg=\Po(\Omega)$ класс $\PPr(\P)$ может вовсе оказаться пустым.}. $\PPr(\P)$~--- класс вероятностей, определённых на $\Alg$, различия которых не могут быть смоделированы в рамках возможностного подхода, и определяемый возможностью $\P$, см. раздел~\ref{sec:poss_model_of_rand} приложения~\ref{sec:optimal_decisions}.

Аналогично вероятностному случаю, рассмотрим серию $L$ независимых испытаний, в которой $l$-ое испытание есть стохастический эксперимент, модель которого~--- вероятностное пространство $\big(\Omega,\, \Alg,\, \Pr^l\big)$ из класса~\eqref{eq:20150821_01}, $l=1,\, \ldots,\, L$. Вероятностной моделью такой серии является вероятностное пространство $\big(\Omega^{(L)},\, \Alg^{(L)},\, \Pr^{(L)}\big)$, где $\Omega^{(L)}$ и $\Alg^{(L)}$ определены в~\eqref{eq:20150818_03}, а вероятность $\Pr^{(L)}$ удовлетворяет условию
$$
    \Pr^{(L)}(A_1\times\ldots\times A_L) = \prod_{l=1}^L \Pr^l(A_l), \quad A_l\in\Po(\Omega),\ l=1,\, \ldots,\, L.
$$

В~\cite{possbook2} показано, что при весьма общих условиях для любых событий ${A,\, B\in\Alg}$ из ${\P(A)>\P(B)}$ при достаточно большом $L$ почти наверное\footnote{Утверждение <<почти наверное>> понимается здесь в терминах вероятностного пространства $\big(\Omega^{(\infty)},\, \Alg^{(\infty)},\, \Pr^{(\infty)}\big) = \big(\Omega,\, \Alg,\, \Pr^1\big)\times\big(\Omega,\, \Alg,\, \Pr^2\big)\times\ldots$} следует ${\nu^{(L)}(A)>\nu^{(L)}(B)}$, что и определяет частотную интерпретацию возможности: чем больше возможность события, тем чаще оно происходит. Частота $\nu^{(L)}(\cdot):\ \Alg\to[0,1]$ определена согласно~\eqref{eq:20150818_04}.



Т.\,в. Ю.~П.~Пытьева построена аксиоматически, аналогично построению теории вероятностей \cite{4,8}. Пусть имеется множество элементарных исходов $\Om$ и задана $\sigma$-алгебра событий $\Alg \subseteq 2^{\Om}$. Возможностью называется всякая измеримая функция $\P(\cdot): \Alg \rightarrow \zo$, обладающая определенными свойствами, которые будут определены ниже. Тройка $\OAP$ называется пространством с возможностью. 

Определим шкалу $\scL$ значений возможности $\P(\cdot)$ как интервал $\zo$ с естественной упорядоченностью $\leqslant$ и двумя бинарными операциями, $\plus: \zo \times \zo \rightarrow \zo$ и $\mult: \zo \times \zo \rightarrow \zo$. С точки зрения эксперта, $\scL$ --- это правила, по которым он размещает возможности событий из $\Alg$ на отрезке $\zo$. 

Пусть $\Gamma$ --- группа строго монотонно возрастающих непрерывных $\gamma(\cdot): \zo \rightarrow \zo,\ \gamma(0)=0,\ \gamma(1)=1$ с групповой операцией $\circ: f \circ g = f(g(\cdot))$. Возможности $\P$ и $\P'$ назовем {\sl эквивалентными}, если $\P' = \gamma(\P)$. Эквивалентные возможности образуют класс возможностей $\PP$ и соотвествующий класс пространств с возможностью \{$\OAP\ |\ \P \in \PP $\}. Принцип относительности (см. \cite{8}) гласит, что утверждения и модели, полученные с использованием возможности, содержательны в том и только в том случае, если они остаются верными при замене данной возможности $\P$ на любую эквивалентную ей $\P'$. 
%Иными словами, не важен выбор конкретной шкалы $\scL$, потому что группа $\Gamma$ порождает группу $\tilde \Gamma$ автоморфизмов $\scL$.

Из принципа относительности следует, что операции $\plus$ и $\mult$ над значениями возможностей двух произвольных множеств $A, B \in \Alg$, т.е. числами $a = \P(A) \in \zo, b = \P(B) \in \zo$, должны удовлетворять условиям:
\begin{center}
  $\gamma(a \plus b) = \gamma(a) \plus \gamma(b),\ $
  $\gamma(a \mult b) = \gamma(a) \mult \gamma(b) $
\end{center}
Также эти операции должны быть коммутативны и не выводить за пределы $\zo$:
\begin{center}
  $a \plus b = b \plus a,\ a \plus 1 = 1,\ a \plus 0 = a, $\\
  $a \mult b = b \mult a,\ a \mult 1 = a,\ a \mult 0 = 0, $
\end{center}
В \cite{8} показано, что единственный способ получить все эти свойства в классе непрерывных операций --- определить
\begin{equation*}
 a \plus b \defeq \sup\{a,b\}; \ \  a \mult b \defeq \inf\{a,b\} 
\end{equation*}
Эти операции естественным образом распространяются на конечное или счётное число операндов. 

Таким образом, возможность --- мера, принимающая значения на шкале $\scL = \scale$, где сложение понимается как точная верхняя грань, а умножение --- как точная нижная грань. По аналогии с вероятностной мерой постулируются два свойства возможности:
\begin{enumerate}
 \item аддитивность: $\P(A \cup B) = \P(A) \plus \P(B)\ \forall A,B \in \Alg$, где $a \plus: b \defeq \sup\{a,b\}, a, b \in \zo$;\footnote{Из-за переопределенной операции сложения в свойстве (1), в отличие от аналогичного свойства вероятностей, не требуется $A \cup B = \o{}$.}
 \item нормировка на единицу: $\P(\Om) = \underset{\omega \in \Om}\sup\ \P({\omega}) = 1$;
\end{enumerate}

Моделью оцениваемого параметра в т.\,в. Пытьева выступает математический объект, аналогичный понятию случайной величины в теории вероятностей --- нечёткий элемент $\tilde x(\cdot): \Om \rightarrow X$. Моделью нечёткой экспертной оценки служит функция распределения возможностей нечёткого элемента $\p_{\tilde x}(\cdot) \defeq \P(\{\tilde x(\omega) = x\})$. 

Итак, принципиальное отличие т.\,в. Пытьева от т.\,в. Заде в том, что в ней содержательно интерпретиуются не все конкретные значения $\p_{\tilde x}(\cdot)$, а только {\sl упорядоченность} этих значений, и отдельно --- неподвижная относительно группы $\Gamma$ точка <<0>>, так что $\p_{\tilde x}(x_0) = 0$ означает принципиальную невозможность значения $x_0 \in X$. Используя такую модель экспертной оценки, можно предоставить эксперту свободу выбора своей субъективной шкалы $\scL$, и экспертные оценки на рис. \ref{ris:fuzzy_ass03} будут выражать одно и то же экспертное мнение.

\begin{figure}[h]
\center{\includegraphics[width=0.7\linewidth]{fuzzy_ass03}}
\caption{\small Пример двух эквивалентных экспертных оценок.}
\label{ris:fuzzy_ass03}
\end{figure}

\end{comment}
